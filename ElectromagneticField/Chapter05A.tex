\section{均匀平面波在理想介质中的传播}

在本节,我们将讨论$\sigma=0$的理想介质中的均匀平面电磁波的传播特性,包括其波函数、波的电磁能量、波速等。而在下一节,我们将会更一般的讨论$\sigma\neq 0$的导电介质中的电磁波。

\subsection{理想介质中的均匀平面波的波函数}
\setpeq{理想介质中的均匀平面波的波函数}

设讨论的区域为$\rho=0$无源区,且充满线性、均匀、各向同性的理想介质,换言之$\sigma=0$且有$\vb*{J}=\sigma\vb*{E}=\vb*{0}$。那时谐电磁场将满足齐次亥姆霍兹方程,根据\fancyref{eqt:亥姆霍兹方程}
\begin{Equation}&[1]
    \laplacian\vb*{\dot{E}}+k^2\dot{\vb*{E}}=\vb*{0}\qquad
    \laplacian\vb*{\dot{H}}+k^2\dot{\vb*{H}}=\vb*{0}
\end{Equation}
将\xrefpeq{1}在直角坐标中展开
\begin{Equation}&[2]
    \qquad\qquad\qquad
    \pdv[2]{\vb*{\dot{E}}}{x}+
    \pdv[2]{\vb*{\dot{E}}}{y}+
    \pdv[2]{\vb*{\dot{E}}}{z}+
    k^2\vb*{\dot{E}}=\vb*{0}\qquad
    \pdv[2]{\vb*{\dot{H}}}{x}+
    \pdv[2]{\vb*{\dot{H}}}{y}+
    \pdv[2]{\vb*{\dot{H}}}{z}+
    k^2\vb*{\dot{H}}=\vb*{0}
    \qquad\qquad\qquad
\end{Equation}
由于我们考虑的是平面电磁波波,而平面波仅沿一个方向传播,不妨设其沿$z$轴方向传播,那么电场强度$\vb*{\dot{E}}$和磁场强度$\vb*{\dot{H}}$都只是坐标$z$的函数,而与坐标$x,y$无关,\xrefpeq{2}即简化为
\begin{Equation}&[3]
    \dv[2]{\vb*{\dot{E}}}{z}+k^2\vb*{\dot{E}}=\vb*{0}\qquad
    \dv[2]{\vb*{\dot{H}}}{z}+k^2\vb*{\dot{H}}=\vb*{0}
\end{Equation}
而另外一方面,依据\fancyref{eqt:麦克斯韦方程组的复数形式}
\begin{Equation}&[4]
    \div\vb*{\dot{E}}=0\qquad
    \div\vb*{\dot{H}}=0
\end{Equation}
将\xrefpeq{4}在直角坐标中展开
\begin{Equation}&[5]
    \pdv{\dot{E_x}}{x}+
    \pdv{\dot{E_y}}{y}+
    \pdv{\dot{E_z}}{z}=0\qquad
    \pdv{\dot{H_x}}{x}+
    \pdv{\dot{H_y}}{y}+
    \pdv{\dot{H_z}}{z}=0
\end{Equation}
由于这里$\vb*{\dot{E}}$和$\vb*{\dot{H}}$仅与$z$有关
\begin{Equation}&[6]
    \dv{\dot{E_z}}{z}=0\qquad
    \dv{\dot{H_z}}{z}=0
\end{Equation}
\xrefpeq{3}和\xrefpeq{6}就是我们求解均匀平面波的波函数的基础了。将\xrefpeq{3}改写为分量形式
\begin{Align}[16pt]
    &\dv[2]{E_x}{z}+kE_x=0&
    &\dv[2]{H_x}{z}+kE_x=0\xlabelpeq{7}\\
    &\dv[2]{E_y}{z}+kE_y=0&
    &\dv[2]{H_y}{z}+kE_y=0\xlabelpeq{8}\\
    &\dv[2]{E_z}{z}+kE_z=0&
    &\dv[2]{H_z}{z}+kE_z=0\xlabelpeq{9}
\end{Align}
然而\xrefpeq{6}指出$\dd{\dot{E_z}}/{\dz}=\dd{\dot{H_z}}/{\dz}=0$,即有$\dd^2{\dot{E_z}}/{\dz^2}=\dd^2{\dot{H_z}}/{\dz^2}=0$,代入\xrefpeq{9}得
\begin{Equation}&[10]
    E_z(z)=0\qquad H_z(z)=0
\end{Equation}
这样还剩下\xrefpeq{7}和\xrefpeq{8},它们是二阶常微分方程,其中$E_x,E_y$的解为
\begin{Gather}[6pt]
    \dot{E_x}(z)=E_{x1\text{m}}\e^{\j\phi_{x1}}\e^{-\j kz}+E_{x2\text{m}}\e^{\j\phi_{x2}}\e^{\j kz}\xlabelpeq{11}\\
    \dot{E_y}(z)=E_{y1\text{m}}\e^{\j\phi_{y1}}\e^{-\j kz}+E_{y2\text{m}}\e^{\j\phi_{y2}}\e^{\j kz}\xlabelpeq{12}
\end{Gather}
这里\xrefpeq{11}和\xrefpeq{12}中的第一项和第二项,分别表示沿$+z$和$-z$传播的平面电磁波。

简单起见,我们这里仅讨论沿$+z$传播的平面电磁波
\begin{Gather}[6pt]
    \dot{E_x}(z)=E_{x\text{m}}\e^{\j\phi_{x}}\e^{-\j kz}\\
    \dot{E_y}(z)=E_{y\text{m}}\e^{\j\phi_{y}}\e^{-\j kz}
\end{Gather}
这样,我们就得到了理想介质中沿$+z$方向传播的平面电磁波的波函数了。
\begin{BoxFormula}[理想介质中平面电磁波的波函数]
    理想介质中,沿$+z$方向传播的平面电磁波的电场强度$\vb*{\dot{\vb*{E}}}(z)$的波函数为
    \begin{Equation}
        \qquad\qquad
        \dot{E_x}(z)=E_{x\text{m}}\e^{\j\phi_x}\e^{-\j kz}\qquad
        \dot{E_y}(z)=E_{y\text{m}}\e^{\j\phi_y}\e^{-\j kz}\qquad
        \dot{E_z}(z)=0
        \qquad\qquad
    \end{Equation}
\end{BoxFormula}
我们或许会问,为什么这里只解了$\vb*{\dot{E}}(z)$而没有以相似方法解出$\vb*{\dot{H}}(z)$?因为根据麦克斯韦方程组中的$\curl\vb*{\dot{E}}=-\j\omega\mu\vb*{\dot{H}}$,磁场强度$\vb*{\dot{H}}(z)$并不独立于电场强度$\vb*{\dot{E}}(z)$,直接写出$\vb*{\dot{H}}(z)$没有实际意义。因此,现在要做的就是在平面电磁波的背景下,建立$\vb*{\dot{H}}(z)$与$\vb*{\dot{E}}(z)$间的关系。\setpeq{理想介质中平面电磁波的电磁矢量关系}

根据\fancyref{eqt:麦克斯韦方程组的复数形式}
\begin{Equation}&[1]
    \vb*{\dot{H}}=-\frac{1}{\j\omega\mu}\curl\vb*{\dot{E}}
\end{Equation}
由于这里$\vb*{\dot{E}}$是仅与$z$有关的函数$\vb*{\dot{E}}(z)$,这里旋度的计算并不复杂
\begin{Equation}&[2]
    \vb*{\dot{H}}=-\frac{1}{\j\omega\mu}\qty(\pdv{E_x}{z}\vb*{e}_y-\pdv{E_y}{z}\vb*{e}_x)
\end{Equation}
代入\fancyref{fml:理想介质中平面电磁波的波函数}
\begin{Equation}&[3]
    \vb*{\dot{H}}=\frac{k}{\omega\mu}\qty(E_x\vb*{e}_y-E_y\vb*{e}_x)
\end{Equation}
代入\fancyref{eqt:亥姆霍兹方程}中$k=\omega\sqrt{\mu\varepsilon}$的定义
\begin{Equation}&[4]
    \vb*{\dot{H}}=\frac{\omega\sqrt{\mu\varepsilon}}{\omega\mu}\qty(E_x\vb*{e}_y-E_y\vb*{e}_x)
    =
    \sqrt{\frac{\varepsilon}{\mu}}\qty(E_x\vb*{e}_y-E_y\vb*{e}_x)
\end{Equation}
我们不妨引入$\eta=\sqrt{\mu/\varepsilon}$作为代换变量
\begin{Equation}&[5]
    \vb*{\dot{H}}=\eta^{-1}\qty(E_x\vb*{e}_y-E_y\vb*{e}_x)
\end{Equation}
使用叉乘表示\xrefpeq{5}的括号项
\begin{Equation}
    \vb*{\dot{H}}=\eta^{-1}\vb*{e}_z\times\vb*{\dot{E}}
\end{Equation}
这样我们就得到了电磁波的电磁矢量间的关系。
\begin{BoxFormula}[电磁波的电磁矢量关系]
    理想介质中,电磁波的电磁矢量间满足以下关系
    \begin{Equation}
        \vb*{\dot{H}}=\eta^{-1}\vb*{e}_z\times\dot{\vb*{E}}\qquad
        \vb*{\dot{E}}=\eta\dot{\vb*{H}}\times\vb*{e}_z
    \end{Equation}
    这里$\eta$称为\uwave{波阻抗}(Wave Impedance)
    \begin{Equation}
        \eta=\frac{|\vb*{\dot{E}}|}{|\vb*{\dot{H}}|}=\sqrt{\frac{\mu}{\varepsilon}}
    \end{Equation}
\end{BoxFormula}
波阻抗是电场与磁场之比,因为具有阻抗的量纲而称为波阻抗。由于波阻抗$\eta$的值与介质的参数有关,因此也称为介质的\uwave{本征阻抗}或\uwave{特性阻抗},特别是,在自由空间中
\begin{Equation}
    \eta_0=\sqrt{\frac{\mu_0}{\varepsilon_0}}=120\pi\approx 377(\si{\ohm})
\end{Equation}
\fancyref{fml:电磁波的电磁矢量关系}指出,在理想介质中的电磁波,电场$\vb*{E}$、磁场$\vb*{H}$、波的传播方向$\vb*{e}_z$之间相互垂直,遵循右手螺旋关系(右手四指由电场转向磁场,右手拇指指向波的传播方向)。至此,我们就可以勾勒出理想介质中的电磁波的物理图景了:电磁波沿$z$方向传播,电场和磁场在垂直于$z$方向的$xy$平面上沿两个相互垂直的方向,作同相位的振动。

\subsection{理想介质中的均匀平面波的能量}\setpeq{理想介质中的均匀平面波的能量}
根据\fancyref{fml:电磁波的电磁矢量关系},由于
\begin{Equation}&[1]
    \vb*{\dot{H}}=\eta^{-1}\vb*{e}_z\times\vb*{\dot{E}}
\end{Equation}
因此
\begin{Equation}&[2]
    |\vb*{\dot{H}}|=\eta^{-1}|\vb*{\dot{E}}|
\end{Equation}
代入$\eta=\sqrt{\mu/\varepsilon}$
\begin{Equation}&[3]
    |\vb*{\dot{H}}|=\sqrt{\frac{\varepsilon}{\mu}}~|\vb*{\dot{E}}|
\end{Equation}
或
\begin{Equation}&[4]
    \frac{1}{4}\varepsilon|\vb*{\dot{E}}|^2=
    \frac{1}{4}\mu|\vb*{\dot{H}}|^2
\end{Equation}
根据\fancyref{fml:平均电场能量密度}和\fancyref{fml:平均磁场能量密度},\xrefpeq{4}的等号两侧恰分别是平均电场能量密度$w_\text{eav}$和平均磁场能量密度$w_\text{mav}$,这表明,\empx{理想介质中,均匀平面电磁波的平均电场能量密度等于平均磁场能量密度},电场和磁场在能量上,是均等的。
\begin{BoxFormula}[理想介质中的均匀平面波的能量]
    理想介质中的均匀平面电磁波,平均电场能量密度等于平均磁场能量密度
    \begin{Equation}
        w_\text{eav}=w_\text{mav}=\frac{1}{4}\varepsilon|\vb*{\dot{E}}|^2
    \end{Equation}
\end{BoxFormula}
根据\fancyref{fml:平均坡印廷矢量}
\begin{Equation}
    \vb*{S}_\text{av}=\frac{1}{2}\Re[\vb*{\dot{E}}\times\vb*{\dot{H}}^{*}]
\end{Equation}
代入\fancyref{fml:电磁波的电磁矢量关系}
\begin{Equation}
    \vb*{S}_\text{av}=\frac{1}{2\eta}\Re\big[\vb*{\dot{E}}\times(\vb*{e}_z\times\vb*{E}^{*})]=\vb*{e}_z\frac{1}{2\eta}|\vb*{\dot{E}}|^2
\end{Equation}
这表明,\empx{理想介质中,均匀平面电磁波的能量沿波的传播方向流动}。
\begin{BoxFormula}[理想介质中均匀平面波的坡印廷矢量]
    理想介质中的均匀平面电磁波,平均坡印廷矢量满足
    \begin{Equation}
        \vb*{S}_\text{av}=\vb*{e}_z\frac{1}{2\eta}|\vb*{\dot{E}}|^2
    \end{Equation}
\end{BoxFormula}

\subsection{相速}\setpeq{相速}
由于前面我们已经证明了当电磁波沿$z$轴传播时,电场和磁场在$xy$平面上沿两个相互垂直的方向振动,如果我们恰当的选取$x$轴和$y$轴,电场和磁场完全可以分别仅在$x$轴和$y$轴存在分量,达成简化分析的目的。在这样的背景下,电场强度的复矢量$\vb*{\dot{E}}(z)$可以表示为
\begin{Equation}&[1]
    \vb*{\dot{E}}(z)=\vb*{e}_xE_{x\text{m}}\e^{\j\phi_x}\e^{-\j kz}
\end{Equation}
如果回到时变形式
\begin{Equation}&[2]
    \vb*{E}(z,t)=\vb*{e}_xE_{x\text{m}}\cos(\omega t-kz+\phi_x)
\end{Equation}
这里$\omega$和$k$的意义就很明确了
\begin{itemize}
    \item 参量$\omega$是电磁波的\uwave{角频率},即单位时间的相位变化。
    \item 参量$k$\hspace{0.3em}是电磁波的\uwave{角波数},即单位空间的相位变化。
\end{itemize}
我们通常所说的电磁波的波速,指的其实是电磁波的等相位面在空间中的移动速度,也称为相速。那么如何定义相速呢,由\xrefpeq{2}可以看出,由于等相位面是恒定值,因此应有
\begin{Equation}
    \omega t-kz=C
\end{Equation}
两边取微分
\begin{Equation}
    \omega\dd{t}-k\dd{z}=0
\end{Equation}
此时,移项作比得到的$\dv*{z}{t}$就可以作为相速的数学定义。
\begin{BoxDefinition}[相速]
    \uwave{相速}(Phase Velocity)定义为等相位面的移动速度
    \begin{Equation}
        v_\text{P}=\dv{z}{t}=\frac{\omega}{k}
    \end{Equation}
\end{BoxDefinition}
在理想介质中,由于$k=\omega\sqrt{\mu\varepsilon}$,因此得到
\begin{BoxFormula}[理想介质中电磁波的相速]
    理想介质中,电磁波的相速为
    \begin{Equation}
        v_\text{P}=\frac{1}{\sqrt{\varepsilon\mu}}
    \end{Equation}
\end{BoxFormula}
在理想介质中,电磁波的相速与频率无关,仅与介质参数有关,特别的,真空中
\begin{Equation}
    v_\text{P0}=\frac{1}{\sqrt{\varepsilon_0\mu_0}}=3\times 10^{8}\si{m\cdot s^{-1}}=c
\end{Equation}
即\uwave{真空光速}(Speed of Light),这也一定程度上说明了光是一种电磁波。