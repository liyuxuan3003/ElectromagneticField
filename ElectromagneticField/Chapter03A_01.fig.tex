\documentclass{xStandalone}

\begin{document}
\begin{tikzpicture}
    \path (0,0) coordinate (O);
    \path (O) ++(6,+2) coordinate (A);
    \path (O) ++(6,-2) coordinate (B);
    
    \fill[red ,fill opacity=0.1] (O) rectangle (A);
    \fill[blue,fill opacity=0.1] (O) rectangle (B);

    \path (O) node[above right=0.2cm] {\small 电介质1};
    \path (O) node[below right=0.2cm] {\small 电介质2};

    \path ($(O)!0.6!(O-|A)$) ++(0,+1) coordinate (P1) node[above right] {$P_1$} node[left] {$\varphi_1$} node[point] {};
    \path ($(O)!0.6!(O-|B)$) ++(0,-1) coordinate (P2) node[below right] {$P_2$} node[left] {$\varphi_2$} node[point] {};

    \draw[ultra thin] (P1) ++(0.5,0) coordinate (P1') -- ++(1,0) coordinate (P1'');
    \draw[ultra thin] (P2) ++(0.5,0) coordinate (P2') -- ++(1,0) coordinate (P2'');

    \draw[ultra thin,<->]  ($(P1')!0.3!(P1'')$) -- node[above right] {$\delt{l}$} ($(P2')!0.3!(P2'')$);
\end{tikzpicture}
\end{document}