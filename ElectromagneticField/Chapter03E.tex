\section{静磁场与电感}

\subsection{电感器和电感}
电感器(Inductor)是广泛使用的电路元件,电感其实就是一个回路(通常会是一个线圈),而实验表明,穿过回路的磁链$\Psi$总是与回路中通过的电流$I$成正比,该比值就是所谓的电感。

\begin{BoxDefinition}[电感]*
    定义\uwave{电感}(Inductance)为电感器的磁链与电流的比值,记作
    \begin{Equation}
        L=\frac{\Psi}{I}
    \end{Equation}
    电感在国际单位制中的单位是\ \si{H}(亨利)。 
\end{BoxDefinition}

\subsection{自感和互感}
为什么通常只有电感会区分\uwave{自感}(Self Inductance)和\uwave{互感}(Mutual Inductance),而电容则通常不会有自容和互容之类的提法?这其实是由电容和电感的定义形式所决定的。

孤立导体的电容其实就是“自容”
\begin{Equation}
    C=\frac{q}{\varphi}
\end{Equation}
孤立回路的电感其实就是“自感”
\begin{Equation}
    L=\frac{\Psi}{I}
\end{Equation}
两个导体构成的系统的电容其实就是“互容”
\begin{Equation}
    C=\frac{q}{\varphi_1-\varphi_2}
\end{Equation}
两个回路构成的系统的电感其实就是“互感”,但且慢,我们怎么定义两个回路间的电感?设有$C_1,C_2$两个回路,记$\Psi_{12}$为$C_1$在$C_2$上产生的磁链,记$\Psi_{21}$为$C_2$在$C_1$上产生的磁链
\begin{Equation}
    M_{12}=\frac{\Psi_{12}}{I_2}\qquad
    M_{21}=\frac{\Psi_{21}}{I_1}
\end{Equation}

这里$M_{12}$和$M_{21}$分别称为“$C_1$对$C_2$的互感”和“$C_2$对$C_1$的互感”,由此可见,\empx{互感衡量的是自身回路的电流能在对方的回路上产生的磁链大小}。这就引出了为何只有电感需要区分自感和互感:就电容而言,从孤立导体到双导体系统,自容是过渡为互容,在该过程中,电势变为了电势差。而有所不同的是,就电感而言,从孤立回路到双回路系统,自感的概念仍然是存在的,互感是与自感同时存在的,因此需要区分两者。此外,互容的值只有一个,互感的值则有两个,定义上看两者未必是相等的。不过幸运的是,两个互感$M_{12},M_{21}$其实是相等的。

\begin{BoxFormula}[纽曼公式]
    两个回路$C_1,C_2$间的互感值相等,且均等于
    \begin{Equation}
        M=\frac{\mu}{4\pi}\Ilot[C_1]\Ilot[C_2]\frac{\dd{\vb*{l}_1}\cdot\dd{\vb*{l}_2}}{\abs{\vb*{r}_1-\vb*{r}_2}}
    \end{Equation}
    该公式称为\uwave{纽曼公式}(Neumann Formula)。
\end{BoxFormula}

\begin{Proof}
    根据\fancyref{fml:磁矢势},由$C_1$在$C_2$上任意一点$\vb*{r}_2$产生的磁矢势
    \begin{Equation}&[1]
        \vb*{A}(\vb*{r}_2)=\frac{\mu}{4\pi}\Ilot[C_1]\frac{I_1\dd{\vb*{l}_1}}{\abs{\vb*{r}_2-\vb*{r}_1}}
    \end{Equation}
    根据\fancyref{fml:磁矢势},由$C_2$在$C_1$上任意一点$\vb*{r}_1$产生的磁矢势
    \begin{Equation}&[2]
        \vb*{A}(\vb*{r}_1)=\frac{\mu}{4\pi}\Ilot[C_2]\frac{I_2\dd{\vb*{l}_2}}{\abs{\vb*{r}_1-\vb*{r}_2}}
    \end{Equation}
    由$C_1$在回路$C_2$上产生的磁链为,应用\fancyref{thm:旋度定理}和\fancyref{def:磁矢势}
    \begin{Equation}&[3]
        \Psi_{12}=\Isot[S_2]\vb*{B}_1\cdot\dd{\vb*{S}_2}=\Ilot[C_2]\curl\vb*{B}_1\cdot\dd{\vb*{l}_2}=\Ilot[C_2]\vb*{A}_1\cdot\dd{\vb*{l}_2}
    \end{Equation}
    由$C_2$在回路$C_1$上产生的磁链为,应用\fancyref{thm:旋度定理}和\fancyref{def:磁矢势}
    \begin{Equation}&[4]
        \Psi_{21}=\Isot[S_1]\vb*{B}_2\cdot\dd{\vb*{S}_1}=\Ilot[C_1]\curl\vb*{B}_2\cdot\dd{\vb*{l}_1}=\Ilot[C_1]\vb*{A}_2\cdot\dd{\vb*{l}_1}
    \end{Equation}
    在\xrefpeq{3}中代入\xrefpeq{1}
    \begin{Equation}&[5]
        \Psi_{12}=\frac{\mu}{4\pi}\Ilot[C_2]\Ilot[C_1]\frac{I_1\dd{\vb*{l}_1}\cdot\dd{\vb*{l}_2}}{\abs{\vb*{r}_2-\vb*{r}_1}}
    \end{Equation}
    在\xrefpeq{4}中代入\xrefpeq{2}
    \begin{Equation}&[6]
        \Psi_{21}=\frac{\mu}{4\pi}\Ilot[C_1]\Ilot[C_2]\frac{I_2\dd{\vb*{l}_2}\cdot\dd{\vb*{l}_1}}{\abs{\vb*{r}_1-\vb*{r}_2}}
    \end{Equation}
    因此$M_{12}$为
    \begin{Equation}&[7]
        M_{12}=\frac{\Psi_{12}}{I_1}=\frac{\mu}{4\pi}\Ilot[C_2]\Ilot[C_1]\frac{\dd{\vb*{l}_1}\cdot\dd{\vb*{l}_2}}{\abs{\vb*{r}_2-\vb*{r}_1}}
    \end{Equation}
    因此$M_{21}$为
    \begin{Equation}&[8]
        M_{21}=\frac{\Psi_{21}}{I_2}=\frac{\mu}{4\pi}\Ilot[C_1]\Ilot[C_2]\frac{\dd{\vb*{l}_2}\cdot\dd{\vb*{l}_1}}{\abs{\vb*{r}_1-\vb*{r}_2}}
    \end{Equation}
    注意到\xrefpeq{7}和\xrefpeq{8}除了顺序略有不同,是相等的,故$M=M_{12}=M_{21}$。
\end{Proof}