\section{亥姆霍兹定理}
矢量场的散度和旋度都是表示矢量场性质的量度,而亥姆霍兹定理之处,矢量场具有的所有性质,都可以由其散度、旋度、边界条件说明。定理的证明比较复杂,这里直接给出结论。
\begin{BoxTheorem}[亥姆霍兹定理]
    在边界为$S$的区域$V$内,矢量场$\vb*{F}$完全由它的散度、旋度、边界条件确定,可以表示为
    \begin{Equation}
        \vb*{F}(\vb*{r})=-\grad u(\vb*{r})+\curl\vb*{A}(\vb*{r})
    \end{Equation}
    其中标势满足
    \begin{Equation}
        u(\vb*{r})=
        \frac{1}{4\pi}
        \Itnt[V]\frac{\grad'\cdot\vb*{F}(\vb*{r'})}{\abs{\vb*{r}-\vb*{r}'}}\dd{V'}-\frac{1}{4}\Isot[S]\frac{\vb*{e}_\text{n}'\cdot\vb*{F}(\vb*{r}')}{\abs{\vb*{r}-\vb*{r}'}}\dd{S'}
    \end{Equation}
    其中矢势满足
    \begin{Equation}
        \vb*{A}(\vb*{r})=
        \frac{1}{4\pi}
        \Itnt[V]\frac{\grad'\times\vb*{F}(\vb*{r'})}{\abs{\vb*{r}-\vb*{r}'}}\dd{V'}-\frac{1}{4}\Isot[S]\frac{\vb*{e}_\text{n}'\times\vb*{F}(\vb*{r}')}{\abs{\vb*{r}-\vb*{r}'}}\dd{S'}
    \end{Equation}
    这就是\uwave{亥姆霍兹定理}或\uwave{亥姆霍兹分解}(Helmholtz Decomposition)。\footnote[2]{从数学形式上,亥姆霍兹分解和傅里叶展开有一定的相似之处(因此称为“分解”也很合理)。}
\end{BoxTheorem}

亥姆霍兹定理指出,矢量场总可以由一个无旋场和一个无散场叠加表示
\begin{itemize}
    \item 无旋场以标势的负梯度呈现,标势由$\vb*{F}$的散度和$\vb*{F}$在边界上的法向分量确定。
    \item 无散场以矢势的正旋度呈现,矢势由$\vb*{F}$的旋度和$\vb*{F}$在边界上的切向分量确定。
\end{itemize}
亥姆霍兹定理有两个值得关注的特殊情况
\begin{itemize}
    \item 若区域$V$内矢量场$\vb*{F}$的散度和旋度处处为零,此时标势$u(\vb*{r})$和矢势$\vb*{A}(\vb*{r})$中的第一项积分为零,这就是说,\empx{无散且无旋的矢量场的性质,完全由矢量场的边界条件决定}。
    \item 若区域$V$是无界的,只要$\vb*{F}$的衰减速度快于$1/\vb*{r}$,第二项积分就为零,此时我们就不再需要考虑边界条件了,因此,\empx{无界空间中矢量场的性质,完全由散度和旋度确定},这其实也说明了,无界空间中不能存在无散且无旋的矢量场,除非这个矢量场是恒零的。
\end{itemize}
亥姆霍兹定理总结了矢量场的基本性质,其意义是非常重要的,故称为\uwave{矢量分析基本定理}。

亥姆霍兹定理的这种想法,其实指导了我们如何去分析一个矢量场,从散度和旋度着手,得到散度方程和旋度方程,组成矢量场基本方程的微分形式。麦克斯韦方程其实就是这样得来的。
