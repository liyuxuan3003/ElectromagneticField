\section{电荷守恒与电流连续性}
电荷周围要产生电场,电流周围要产生磁场,电荷电流是产生电磁场的源量。

\subsection{电荷与电荷密度}
自然界中存在两种\uwave{电荷}(Electric Charge):\uwave{正电荷}和\uwave{负电荷}。带电体所带电量称为\uwave{电荷量},电荷量并非是连续变化的,电荷量只能取值基本电荷$e=1.602\times 10^{-19}\si{C}$的整数倍,即,电荷是以离散方式分布的。基本电荷$e$恰好就是质子和电子的电荷量,质子带电$e$,电子带电$-e$。

电荷通常以$q$表示,尽管我们说,电荷实际上是离散分布的,但在研究宏观电磁现象时,我们所观察到的是带电体上大量微观带电粒子的总体效应,而带电粒子的线度又远小于带电体的线度。因此,我们可以近似认为,电荷是连续分布在带电体上的,并可以用电荷密度来描述。

\begin{BoxDefinition}[电荷体密度]*
    连续分布在体积$V$内的电荷,称为\uwave{体分布电荷}或\uwave{体电荷}(Volume Charge)。

    体电荷的分布,用\uwave{电荷体密度}(Volume Charge Density)表示
    \begin{Equation}
        \rho(\vb*{r},t)=\Lim[\delt{V}\to 0]\frac{\delt{q}(\vb*{r},t)}{\delt{V}}=\dv{q(\vb*{r},t)}{V}
    \end{Equation}
    体积$V$内的电荷量可以用电荷体密度$\rho(\vb*{r},t)$在$V$内的积分求出
    \begin{Equation}
        q(t)=\Itnt[V]\rho(\vb*{r},t)\dd{V}
    \end{Equation}
\end{BoxDefinition}

除了电荷体密度,亦有适用于面分布的面密度和线分布的线密度。

\begin{BoxDefinition}[电荷面密度]
    连续分布在曲面$S$内的电荷,称为\uwave{面分布电荷}或\uwave{面电荷}(Surface Charge)。

    面电荷的分布,用\uwave{电荷面密度}(Surface Charge Density)表示
    \begin{Equation}
        \rho_S(\vb*{r},t)=\Lim[\delt{S}\to 0]\frac{\delt{q}(\vb*{r},t)}{\delt{S}}=\dv{q(\vb*{r},t)}{S}
    \end{Equation}
    曲面$S$内的电荷量可以用电荷面密度$\rho_S(\vb*{r},t)$在$V$内的积分求出
    \begin{Equation}
        q(t)=\Isnt[S]\rho_S(\vb*{r},t)\dd{S}
    \end{Equation}
\end{BoxDefinition}

\begin{BoxDefinition}[电荷线密度]
    连续分布在曲线$l$内的电荷,称为\uwave{线分布电荷}或\uwave{线电荷}(Linear Charge)。

    线电荷的分布,用\uwave{电荷线密度}(Linear Charge Density)表示
    \begin{Equation}
        \rho_l(\vb*{r},t)=\Lim[\delt{l}\to 0]\frac{\delt{q}(\vb*{r},t)}{\delt{l}}=\dv{q(\vb*{r},t)}{l}
    \end{Equation}
    曲线$l$内的电荷量可以用电荷线密度$\rho_l(\vb*{r},t)$在$l$内的积分求出
    \begin{Equation}
        q(t)=\Int[l]\rho_l(\vb*{r},t)\dd{l}
    \end{Equation}
\end{BoxDefinition}

当带电体的线度,远小于观察点至带电体的距离时,此时,带电体的形状和电荷分布以及无关紧要,带电体可以近似为一个电荷集中在中心的几何点模型,称为\uwave{点电荷}(Point Charge)。

点电荷的概念在电磁理论中有很重要的地位,就像质点与之经典力学那般。

点电荷是电荷分布的一种极限情况,点电荷可以视为一个体积很小而带电量一定的带电球体在半径$a\to 0$时的极限。设电荷$q(t)$分布在位于$\vb*{r}'$半径为$a$的球体内,那么
\begin{itemize}
    \item 在$|\vb*{r}-\vb*{r}'|>a$的球外区域,电荷密度显然为零。
    \item 在$|\vb*{r}-\vb*{r}'|<a$的球内区域,电荷密度将为非常大的数值(由于球体很小),当$a\to 0$时球心$\vb*{r}=\vb*{r'}$处的电荷密度将趋于无穷大,而$\vb*{r}\neq\vb*{r}'$处的电荷密度则均为零,但整个极限过程中,电荷总量保持为$q(t)$不变。这也就是说,电荷密度$\rho(\vb*{r},t)$在全空间的积分是定值,在$\vb*{r}=\vb*{r}'$为无穷大,在$\vb*{r}\neq\vb*{r}'$为零,这在数学上完全符合狄拉克函数的特性。
\end{itemize}
因此,点电荷的电荷密度可以用狄拉克函数描述。
\begin{BoxDefinition}[点电荷]
    点电荷的电荷密度可以用狄拉克函数描述
    \begin{Equation}
        \rho(\vb*{r},t)=q(t)\dirac(\vb*{r}-\vb*{r}')
    \end{Equation}
    其中,$q(t)$是点电荷的电荷量,$\vb*{r}'$是点电荷的位矢。
\end{BoxDefinition}

\subsection{电流与电流密度}\vspace{-0.25cm}
\begin{BoxDefinition}[电流]
    \uwave{电流}(Electric Current)是由电荷运动形成的,定义为
    \begin{Equation}
        i(t)=\Lim[\delt{t}\to 0]\frac{\delt{q}(t)}{\delt{t}}=\dv{q(t)}{t}
    \end{Equation}
    其中,$\delt{q}(t)$表示在$\delt{t}$的时间内通过某一曲面$S$的电荷量,$i(t)$即通过曲面$S$的电流。
\end{BoxDefinition}

电流的意义是单位时间通过某一曲面$S$的电荷量,因此,我们不能说某一点的电流是多少,但是,我们又迫切的需要了解电流的分布情况,介于不同点处的电荷运动方向往往是不同的。

电流密度$\vb*{J}(\vb*{r},t)$就是为此而引入的,在$\vb*{r}$处的电流密度是如何计算的呢?我们在$\vb*{r}$处垂直电荷运动的方向上取一个面积元$\delt{S}$,其法向单位矢量为$\vb*{e}_\text{n}$,电流密度的大小即该面积元上的电流$\delt{i}$与该面积元$\delt{S}$的比在$\delt{S}\to 0$的极限,电流密度的方向就是该处$\vb*{e}_\text{n}$的方向。

\begin{BoxDefinition}[电流密度]
    \uwave{电流密度}(Current Density)描述了电流的分布
    \begin{Equation}
        \vb*{J}(\vb*{r},t)=\vb*{e}_\text{n}\Lim[\delt{S}\to 0]\frac{\delt{i}(\vb*{r})}{\delt{S}}=\vb*{e}_\text{n}\dv{i(\vb*{r})}{S}
    \end{Equation}
    电流则是电流密度的通量,通过曲面$S$的电流可以被表示为
    \begin{Equation}
        i(t)=\Isnt[S]\vb*{J}(\vb*{r},t)\cdot\dd{\vb*{S}}
    \end{Equation}
\end{BoxDefinition}
\begin{BoxFormula}[电流密度与电荷密度]
    电流密度,是电荷密度与电荷运动速度的乘积
    \begin{Equation}
        \vb*{J}(\vb*{r},t)=\rho\vb*{v}
    \end{Equation}
\end{BoxFormula}\nopagebreak

上述的电流体密度$\vb*{J}(\vb*{r},t)$描述的是体分布电流,我们还可以定义电流面密度$\vb*{J}_S(\vb*{r},t)$来描述面分布电流。但是,我们没有“电流线密度”的概念,因为“电流线密度”就是“电流$i$”,这种线电流的模型是很常用的,例如导线的横截面相较其长度而言很小,就可以视为线电流。\goodbreak

在分析电磁场时,常会使用\uwave{电流元}(Current Element)的概念,对于线电流,我们会沿电流流动方向取一个线元矢量$\dd{\vb*{l}}$,并将$i\dd{\vb*{l}}$称为电流元。而对于体分布电流和面分布电流,其电流元分别是$\vb*{J}\dd{V}$和$\vb*{J}_S\dd{S}$。在某种意义上,载流导体可以视为由电流元“积分”而来的。


\subsection{电荷守恒定律与电流连续性方程}
电荷是守恒的,它既不能被创造,也不能被消灭,只能从物体的一个部分转移到另一部分,或者从一个物体转移到另一个物体,这就是所谓\uwave{电荷守恒定律}(Law of Charge Conservation)。

电荷守恒定律的直接结果,就是\uwave{电流连续性方程}(Continuity Equation)。

\begin{BoxEquation}[电流连续性方程]
    电流连续性方程是指,电流场$\vb*{J}(\vb*{r},t)$满足以下方程
    \begin{Equation}
        \div\vb*{J}+\pdv{\rho}{t}=0
    \end{Equation}
    特别的,对于恒定电流场
    \begin{Equation}
        \div{\vb*{J}}=0
    \end{Equation}
\end{BoxEquation}
\begin{Proof}
    任取一个边界曲面为$S$的空间区域$V$,应有
    \begin{Equation}&[1]
        \Isot[S]{\vb*{J}\cdot\dd{\vb*{S}}}=-\dv{q}{t}=-\dv{t}\Itnt[V]\rho\dd{V}
    \end{Equation}
    这是因为,\xrefpeq{1}左端代表的是单位时间内从$S$流出的电荷,\xrefpeq{1}右端代表的是单位时间在空间区域$V$内减少的电荷,根据前述的电荷守恒定律,这两者很明显应该是相等的。

    由于$V$不会随时间变化,不妨将\xrefpeq{1}右端对时间的求导置于积分内
    \begin{Equation}&[2]
        \Isot[S]\vb*{J}\cdot\dd{\vb*{S}}=-\Itnt[V]\pdv{\rho}{t}\dd{V}
    \end{Equation}
    就\xrefpeq{2}的左端运用\fancyref{thm:散度定理}
    \begin{Equation}
        \Itnt[V]\div\vb*{J}\dd{V}=-\Itnt[V]\pdv{\rho}{t}\dd{V}
    \end{Equation}
    即
    \begin{Equation}
        \Itnt[V]\qty(\div\vb*{J}+\pdv{\rho}{t})\dd{V}=0
    \end{Equation}
    由于$V$是任取的
    \begin{Equation}
        \div\vb*{J}+\pdv{\rho}{t}=0
    \end{Equation}
    而恒定电流场中,电荷分布$\rho$不随时间变化,因此$\pdv*{\rho}{t}=0$。
\end{Proof}
电流连续性方程指出,\empx{时变电流场是有散场}
\begin{itemize}
    \item 电流场在电荷密度随时间减少的地方,发出电流线,构成电流场的源。
    \item 电流场在电荷密度随时间增加的地方,终止电流线,构成电流场的汇。
\end{itemize}
恒定电流场的电荷密度不随时间变化,因此没有源或汇,换言之,\empx{恒定电流场是无散场}。

