\section{静磁场的分析}

\subsection{磁矢势的定义}
根据\fancyref{ppt:静磁场的散度},磁感强度$\vb*{B}$是无散的
\begin{Equation}
    \div\vb*{B}=0
\end{Equation}
根据\fancyref{def:无散场的矢势},无散场可以定义矢势,现在,我们定义$\vb*{B}$的矢势。

\begin{BoxDefinition}[磁矢势]
    定义磁感强度$\vb*{B}$的矢势为\uwave{磁矢势}(Magnetic Vector Potential),记作
    \begin{Equation}
        \vb*{B}(\vb*{r})=\curl\vb*{A}(\vb*{r})
    \end{Equation}
    这就是说,磁矢势的旋度是磁感强度。
\end{BoxDefinition}
诚然,如果我们能完全从矢量分析的角度(这是极好的),通过无旋场的标势和无散场的矢势的观点来理解电势$\varphi(\vb*{r})$和磁矢势$\vb*{A}(\vb*{r})$,那两者是完全对应的。但是,我们经常会有这样的困惑,势往往与能量相关,\empx{电势可以诠释为“每单位电荷储存的能量”},磁矢势却好像没有代表什么能量,事实上在早期,磁矢势确实被认为没有明确的物理意义,但詹姆斯·麦克斯韦对此却颇不以为然\cite{W5},它认为,\empx{磁矢势可以诠释为“每单位电荷储存的动量”},这就对应了。

磁矢势和电势一样,都不能由其定义完全确定,但不同的是,电势具有的是一个任意常数的不确定性,磁矢势具有的则是叠加一个任意无旋场的不确定性,试想,若
\begin{Equation}
    \curl\vb*{A}=\vb*{B}
\end{Equation}
那么,必然同时有
\begin{Equation}
    \curl(\vb*{A}+\grad\psi)=\vb*{B}
\end{Equation}
这是因为梯度场$\grad\psi$必然是一个无旋场,对旋度的计算无影响。

根据\fancyref{thm:亥姆霍兹定理},唯一确定一个矢量场,需要同时给出其旋度和散度,磁矢势的旋度是被定义确定的,因此,还需要规定磁矢势的散度,在静磁场中,常使用的是以下规范。
\begin{BoxDefinition}[磁矢势的库伦规范]
    在静磁场中,常令磁矢势遵循\uwave{库伦规范}(Coulomb Gauge),即
    \begin{Equation}
        \div\vb*{A}=0
    \end{Equation}
\end{BoxDefinition}

电势通过规定参考点就可以唯一确定,磁矢势则需要通过规定散度唯一确定,很明显,后者要比前者复杂很多,为何有这样的差异?这就是因为磁矢势是矢量场,而矢量的任意性强很多。

\subsection{磁矢势的微分方程}
在静磁场中,我们往往会将磁场的矢量方程及其边界条件
\begin{Gather}[6pt]
    \curl\vb*{H}=\vb*{J}\qquad\vb*{e}_\text{n}\times(\vb*{H}_1-\vb*{H}_2)=\vb*{J}_S\\
    \div\vb*{B}=0\qquad\vb*{e}_\text{n}\cdot(\vb*{B}_1-\vb*{B}_2)=0
\end{Gather}
转化为关于磁矢势$\vb*{A}(\vb*{r})$的矢量方程。

\begin{BoxEquation}[静磁场的微分方程]
    静磁场可以有磁矢势的泊松方程描述
    \begin{Equation}&[A]
        \laplacian\vb*{A}=-\mu\vb*{J}(\vb*{r})
    \end{Equation}
    边界条件满足
    \begin{Equation}&[B]
        \vb*{A}_1=\vb*{A}_2\qquad
        \vb*{e}_\text{n}\times\qty(\frac{1}{\mu_1}\curl\vb*{A}_1-\frac{1}{\mu_2}\curl\vb*{A}_2)=\vb*{J}_S
    \end{Equation}
\end{BoxEquation}

\begin{Proof}
    根据\fancyref{eqt:麦克斯韦方程组}和\fancyref{def:磁矢势}
    \begin{Equation}
        \vb*{J}=\curl\vb*{H}=\curl(\mu^{-1}\vb*{B})=\mu^{-1}\curl(\curl\vb*{A})
    \end{Equation}
    即
    \begin{Equation}
        \curl\curl\vb*{A}=\mu\vb*{J}
    \end{Equation}
    根据\fancyref{fml:矢量拉普拉斯的转化}
    \begin{Equation}
        \grad(\div\vb*{A})-\laplacian\vb*{A}=\mu\vb*{J}
    \end{Equation}
    根据\fancyref{def:磁矢势的库伦规范},由于$\div\vb*{A}=0$
    \begin{Equation}
        \laplacian\vb*{A}=-\mu\vb*{J}
    \end{Equation}
    下面我们来将$\vb*{B},\vb*{H}$的边界条件转化为$\vb*{A}$的边界条件,这需要一定的矢量分析。

    根据\fancyref{fml:磁感应强度的边界条件}
    \begin{Equation}
        \vb*{e}_\text{n}\cdot(\vb*{B}_1-\vb*{B}_2)=0
    \end{Equation}
    代入\fancyref{def:磁矢势}
    \begin{Equation}
        \vb*{e}_\text{n}\cdot(\curl\vb*{A}_1)=\vb*{e}_\text{n}\cdot(\curl\vb*{A}_2)
    \end{Equation}
    运用\fancyref{fml:标量三重积的轮换对称性}
    \begin{Equation}
        \div(\vb*{A}_1\times\vb*{e}_\text{n})=
        \div(\vb*{A}_2\times\vb*{e}_\text{n})
    \end{Equation}
    运用乘积法则
    \begin{Equation}
        \qquad\qquad\qquad\qquad
        (\div\vb*{A}_1)\times\vb*{e}_\text{n}+\vb*{A}_1\times(\div\vb*{e}_\text{n})=
        (\div\vb*{A}_2)\times\vb*{e}_\text{n}+\vb*{A}_2\times(\div\vb*{e}_\text{n})
        \qquad\qquad\qquad\qquad
    \end{Equation}
    根据\fancyref{def:磁矢势的库伦规范},这里$\div\vb*{A}_1=\div\vb*{A}_2=0$
    \begin{Equation}
        \vb*{A}_1\times(\div\vb*{e}_\text{n})=\vb*{A}_2\times(\div\vb*{e}_\text{n})
    \end{Equation}
    这里的$\div\vb*{e}_\text{n}$可以被约去,因此在边界两侧
    \begin{Equation}
        \vb*{A}_1=\vb*{A}_2
    \end{Equation}
    根据\fancyref{fml:磁场强度的边界条件}
    \begin{Equation}
        \vb*{e}_\text{n}\times(\vb*{H}_1-\vb*{H}_2)=\vb*{J}_S
    \end{Equation}
    运用介质方程
    \begin{Equation}
        \vb*{e}_\text{n}\times\qty(\frac{1}{\mu}\vb*{B}_1-\frac{1}{\mu}\vb*{B}_2)=\vb*{J}_S
    \end{Equation}
    代入\fancyref{def:磁矢势}
    \begin{Equation}
        \vb*{e}_\text{n}\times\qty(\frac{1}{\mu_1}\curl\vb*{A}_1-\frac{1}{\mu_2}\curl\vb*{A}_2)=\vb*{J}_S
    \end{Equation}
    由此就解得了\xrefpeq{B}的边界条件。
\end{Proof}

\subsection{磁矢势的计算}
\begin{BoxFormula}[磁矢势]
    磁矢势$\vb*{A}(\vb*{r})$符合以下规律
    \begin{Equation}
        \vb*{A}(\vb*{r})=\frac{\mu}{4\pi}\Itnt[V]\frac{\vb*{J}}{R}\dd{V'}
    \end{Equation}
\end{BoxFormula}

\begin{Proof}
    根据\fancyref{eqt:静磁场的微分方程}
    \begin{Equation}
        \laplacian\vb*{A}=-\mu\vb*{J}
    \end{Equation}
    在直角坐标系中,上式的矢量拉普拉斯可以分解为三个标量拉普拉斯
    \begin{Gather}[6pt]
        \laplacian A_x=-\mu J_x\\
        \laplacian A_y=-\mu J_y\\
        \laplacian A_z=-\mu J_z
    \end{Gather}
    这三个分量式与\fancyref{eqt:静电场的微分方程}是类似的
    \begin{Equation}
        \laplacian\varphi=-\frac{\rho}{\varepsilon}
    \end{Equation}
    我们已知电势满足\fancyref{fml:电势}
    \begin{Equation}
        \varphi(\vb*{r})=\frac{1}{4\pi\varepsilon}\Itnt[V]\frac{\rho(\vb*{r}')}{R}\dd{V'}
    \end{Equation}
    我们因此可以从微分式的相似性,推定磁矢势的各分量满足
    \begin{Gather}[12pt]
        A_x=\frac{\mu}{4\pi}\Itnt[V]\frac{J_x(\vb*{r'})}{R}\dd{V'}\\
        A_y=\frac{\mu}{4\pi}\Itnt[V]\frac{J_y(\vb*{r'})}{R}\dd{V'}\\
        A_z=\frac{\mu}{4\pi}\Itnt[V]\frac{J_z(\vb*{r'})}{R}\dd{V'}
    \end{Gather}
    而将上面三个式子叠加在一起,即得
    \begin{Equation}*
        \vb*{A}(\vb*{r})=\frac{\mu}{4\pi}\Itnt[V]\frac{\vb*{J}}{R}\dd{V'}\qedhere
    \end{Equation}
\end{Proof}

\subsection{磁标势的定义}
通常来说,磁场是不可以定义标势的,因为磁场是有旋的
\begin{Equation}
    \curl\vb*{H}=\vb*{J}
\end{Equation}
但是,如果我们所研究的空间不存在自由电流,即有$\vb*{J}=\vb*{0}$,那么
\begin{Equation}
    \curl\vb*{H}=0
\end{Equation}
此时我们就可以定义所谓的磁标势,由此可见,磁标势只是特定情况下可以存在的概念。
\begin{BoxDefinition}[磁标势]
    定义磁场强度$\vb*{H}$在标势为\uwave{磁标势}(Magnetic Scalar Potential),记作
    \begin{Equation}
        \vb*{H}(\vb*{r})=-\grad\psi_\text{m}
    \end{Equation}
    这就是说,磁标势的负梯度是磁场强度。

    该定义仅成立于$\vb*{J}=\vb*{0}$的空间区域,对于$\vb*{J}\neq\vb*{0}$的空间区域不适用。
\end{BoxDefinition}
