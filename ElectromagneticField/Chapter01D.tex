\section{矢量场的性质}
若物理量是一个矢量,则该物理量所确定的场称为\uwave{矢量场}(Vector Field),常用$\vb*{F}$表示
\begin{Equation}
    \vb*{F}=\vb*{F}(\vb*{r},t)
\end{Equation}
而对于时不变的矢量场,则可以表示为
\begin{Equation}
    \vb*{F}=\vb*{F}(\vb*{r})
\end{Equation}
矢量场总是可以分解为三个分量场,在直角坐标系中
\begin{Equation}
    \vb*{F}(\vb*{r})=\vb*{e}_xF_x(\vb*{r})+\vb*{e}_yF_y(\vb*{r})+\vb*{e}_zF_z(\vb*{r})
\end{Equation}
在本节,我们就将研究矢量场的性质。

\subsection{矢量场的矢量线}
矢量场应该如何图示化呢?答案是通过一系列的有向曲线来描述矢量的分布,我们期望
\begin{itemize}
    \item 这些有向曲线上任一点的切线方向,表示该点处$\vb*{F}(r)$的方向。
    \item 这些有向曲线间的疏密,表示该处$\vb*{F}(r)$模值的大小。
\end{itemize}
这样的曲线称为矢量场的矢量线,那么矢量线应当如何从数学上确定呢?其实很简单,有向曲线的切线如果要表示$\vb*{F}$的增加方向,就是要使得矢量线在该处的微元$\dd{\vb*{r}}$与$\vb*{F}$平行,即
\begin{BoxDefinition}[矢量场的矢量线]
    矢量场的可视化方法是\uwave{矢量线}(Streamline),其是由微分方程
    \begin{Equation}
        \dd{\vb*{r}}\times\vb*{F}=\vb*{0}
    \end{Equation}
    特别的,在直角坐标系中
    \begin{Equation}
        \frac{\dx}{F_x}=\frac{\dy}{F_y}=\frac{\dz}{F_z}
    \end{Equation}
    所确定的曲线族,称为\uwave{矢量线族},该方程也称为\uwave{矢量线方程}。
\end{BoxDefinition}

\subsection{矢量场的通量和散度}
\begin{BoxDefinition}[矢量场的通量]
    定义矢量场$\vb*{F}(\vb*{r})$在闭合空间曲面$S$上的\uwave{通量}(Flux)
    \begin{Equation}
        \Phi=\Isot[S]\vb*{F}\cdot\dd{\vb*{S}}
    \end{Equation}
\end{BoxDefinition}
\begin{BoxDefinition}[矢量场的散度]
    定义矢量场$\vb*{F}(\vb*{r})$的\uwave{散度}(Divergence),为该点处的通量密度
    \begin{Equation}
        \div\vb*{F}=\Div\vb*{F}=\Lim[\delt{V}\to 0]\frac{1}{\delt{V}}\Isot[S]\vb*{F}\cdot\dd{\vb*{S}}
    \end{Equation}
    其中$S$是包围点$\vb*{r}$的任意闭曲面,$\delt{V}$是其包围体积,$\delt{V}\to 0$表示曲面收缩至$\vb*{r}$点。
\end{BoxDefinition}
散度是一个标量场,其表征了通量密度,因此
\begin{itemize}
    \item 若散度$\div\vb*{F}>0$,则该点为正通量源(源),矢量线由该点发出。
    \item 若散度$\div\vb*{F}<0$,则该点为负通量源(汇),矢量线向该点汇聚。
\end{itemize}
散度在各个坐标系中的形式,我们在数学物理方法中已经非常熟悉了
\begin{BoxFormula}[直角坐标系的散度]
    在直角坐标系下,散度的形式是
    \begin{Equation}
        \div\vb*{F}=\pdv{F_x}{x}+\pdv{F_y}{y}+\pdv{F_z}{z}
    \end{Equation}
\end{BoxFormula}

\begin{BoxFormula}[柱坐标系的散度]
    在柱坐标系下,散度的形式是
    \begin{Equation}
        \div\vb*{F}=\frac{1}{\rho}\pdv{(\rho F_\rho)}{\rho}+\frac{1}{\rho}\pdv{(F_\phi)}{\phi}+\pdv{F_z}{z}
    \end{Equation}
\end{BoxFormula}

\begin{BoxFormula}[球坐标系的散度]
    在球坐标系下,散度的形式是
    \begin{Equation}
        \div\vb*{F}=\frac{1}{r^2}\pdv{(r^2F_r)}{r}+\frac{1}{r\sin\theta}\pdv{(\sin\theta F_\theta)}{\theta}+\frac{1}{r\sin\theta}\pdv{F_\phi}{\phi}
    \end{Equation}
\end{BoxFormula}

散度的一个重要结论是散度定理,我们在微积分2中已经学过了
\begin{BoxTheorem}[散度定理]
    \uwave{散度定理}(Divergence Theorem),亦称\uwave{高斯定理}(Gauss's Theorem),是指
    \begin{Equation}
        \Itnt[V]\div\vb*{F}\dd{V}=\Isot[S]\vb*{F}\cdot\dd{\vb*{S}}
    \end{Equation}
    其指出,矢量场的散度在$V$上的积分等于矢量场在$V$的边界曲面$S$上的通量。
\end{BoxTheorem}

\subsection{矢量场的环流和旋度}
通量由矢量函数的在闭曲面上的积分给出,环流则由矢量函数在闭取向上的积分给出。
\begin{BoxDefinition}[矢量场的环流]
    定义矢量场$\vb*{F}(\vb*{r})$在闭合空间曲线$C$上的环流(Circulation)
    \begin{Equation}
        \Gamma=\Ilot[C]\vb*{F}\cdot\dd{\vb*{l}}
    \end{Equation}
\end{BoxDefinition}

通量密度是散度,环流密度也就是所谓的旋度,但是,这里略微有些不一样的地方,试对比
\begin{Equation}
    \Lim[\delt{V}\to 0]\frac{1}{\delt{V}}\Isot[S]\vb*{F}\cdot\dd{\vb*{S}}\qquad\Lim[\delt{S}\to 0]\frac{1}{\delt{S}}\Ilot[C]\vb*{F}\cdot\dd{\vb*{l}}
\end{Equation}
通过通量密度定义散度时,当$\delt{V}\to 0$时曲面$S$总会缩至一点,无论闭曲面$S$以何种方式收缩,结果都完全相同的。但是,通过环流密度试定义旋度时,当$\delt{S}\to 0$时曲线$C$虽然也会缩至一点,但是取决于面元$\delt{S}$的取向,该极限可能会有不同的结果,我们姑且可以将其称为“方向旋度”。正如从“方向导数”到梯度,也可以类似的从“方向旋度”到旋度。即对每个点,选取该点处方向旋度的最大值作为模长,选取此时面积元$\delt{S}$的法向矢量作为方向。

因此,尽管环流密度是标量,旋度却应该是矢量,旋度的正式定义如下
\begin{BoxDefinition}[旋度]
    定义矢量场$\vb*{F}(\vb*{r})$的旋度(Curl),为该点处的环流密度\footnote[2]{这是简单的提法,并不准确。}
    \begin{Equation}
        \curl F=\Curl F=\vb*{e}_\text{n}\max\Lim[\delt{S}\to 0]\frac{1}{\delt{S}}\Ilot[C]\vb*{F}\cdot\dd{\vb*{l}}
    \end{Equation}
    其中$C$是包围点$\vb*{r}$的任意闭取向,$\delt{S}$是其包围面积,$\delt{S}\to 0$表示取向收缩至$\vb*{r}$点。

    其中,$\vb*{e}_\text{n}$表示使得环流密度取值最大的那一$\delt{S}$的法向单位矢量。
\end{BoxDefinition}
旋度是一个矢量场,其模表示最大环流密度,其方向代表最大环流密度的朝向。

旋度也可以被认为是某种源,只不过说
\begin{itemize}
    \item 散度的源导致通量的产生,称为\uwave{通量源},它导致矢量线的发出与汇聚。
    \item 旋度的源导致环流的产生,称为\uwave{涡旋源},它导致矢量线的正旋与逆旋。
\end{itemize}
这样,我们就可以说
\begin{itemize}
    \item 若旋度$\curl\vb*{F}>0$,则该点为正涡旋源(正旋),矢量线绕该点以旋度方向右手螺旋。
    \item 若旋度$\curl\vb*{F}<0$,则该点位负涡旋源(逆旋),矢量线绕该点以旋度方向左手螺旋。
\end{itemize}
旋度在各个坐标系中的形式,我们在数学物理方法中已经非常熟悉了
\begin{BoxFormula}[直角坐标系的旋度]
    在直角坐标系下,旋度的形式是
    \begin{Equation}
        \curl\vb*{F}=
        \begin{vmatrix}
            \vb*{e}_x&\vb*{e}_y&\vb*{e}_z\\[1mm]
            \pdv*{x}&\pdv*{y}&\pdv*{z}\\[1mm]
            F_x&F_y&F_z\\
        \end{vmatrix}
    \end{Equation}    
\end{BoxFormula}

\begin{BoxFormula}[柱坐标系的旋度]
    在柱坐标系下,旋度的形式是
    \begin{Equation}
        \curl\vb*{F}=
        \begin{vmatrix}
            \vb*{e}_\rho&\rho\vb*{e}_\phi&\vb*{e}_z\\[1mm]
            \pdv*{\rho}&\pdv*{\phi}&\pdv*{z}\\[1mm]
            F_\rho&\rho F_\rho&F_z\\
        \end{vmatrix}
    \end{Equation}    
\end{BoxFormula}

\begin{BoxFormula}[球坐标系的旋度]
    在球坐标系下,旋度的形式是
    \begin{Equation}
        \curl\vb*{F}=
        \begin{vmatrix}
            \vb*{e}_r&r\vb*{e}_\theta&r\sin\theta\vb*{e}_\phi\\[1mm]
            \pdv*{r}&\pdv*{\theta}&\pdv*{\phi}\\[1mm]
            F_r&rF_\theta&r\sin\theta F_\phi\\
        \end{vmatrix}
    \end{Equation}    
\end{BoxFormula}

旋度的一个重要结论是旋度定理,我们在微积分2中已经学过了
\begin{BoxTheorem}[旋度定理]
    \uwave{旋度定理}(Curl Theorem),亦称\uwave{斯托克斯定理}(Stokes Theorem),是指
    \begin{Equation}
        \Isnt[S]\curl\vb*{F}\cdot\dd{\vb*{S}}=\Ilot[C]\vb*{F}\cdot\dd{\vb*{l}}
    \end{Equation}
    其指出,矢量场的旋度在$S$上的积分等于矢量场在$S$的边界曲线$C$上的环流。
\end{BoxTheorem}
