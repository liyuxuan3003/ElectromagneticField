\section{矢量代数}
有关\uwave{标量}(Scalar)和\uwave{矢量}(Vector)的定义和矢量的一些基本性质,我们在高等数学中已经很熟悉了,这里不再赘述。这里重点复习一下矢量的两种乘法,即点积和叉积的定义和性质。

\subsection{矢量的点积}
\begin{BoxDefinition}[点积]
    矢量的\uwave{点积}(Dot Product)是一个标量,定义为
    \begin{Equation}
        \vb*{A}\cdot\vb*{B}=AB\cos\theta
    \end{Equation}
\end{BoxDefinition}
\begin{BoxProperty}[点积的性质]
    点积满足交换律
    \begin{Equation}
        \vb*{A}\cdot\vb*{B}=\vb*{B}\cdot\vb*{A}
    \end{Equation}
    点积满足分配律
    \begin{Equation}
        \vb*{A}\cdot(\vb*{B}+\vb*{C})=\vb*{A}\cdot\vb*{B}+\vb*{A}\cdot\vb*{C}
    \end{Equation}
\end{BoxProperty}

\subsection{矢量的叉积}
\begin{BoxDefinition}[叉积]
    矢量的\uwave{叉积}(Cross Product)是一个矢量,定义为
    \begin{Equation}
        \vb*{A}\times\vb*{B}=AB\sin\theta\vb*{e}_\text{n}
    \end{Equation}
    其中$\vb*{e}_n$是右手由矢量$\vb*{A}$选择至$\vb*{B}$时,大拇指的方向。
\end{BoxDefinition}
\begin{BoxProperty}[叉积的性质]
    叉积满足反交换律
    \begin{Equation}
        \vb*{A}\times\vb*{B}=-\vb*{B}\times\vb*{A}
    \end{Equation}
    叉积满足分配律
    \begin{Equation}
        \vb*{A}\times(\vb*{B}+\vb*{C})=\vb*{A}\times\vb*{B}+\vb*{A}\times\vb*{C}
    \end{Equation}
\end{BoxProperty}

\subsection{矢量的三重积}
矢量有点积和叉积两种“相乘”运算,因此当三个矢量“相乘”时,会出现两种三重积。\cite{W1}

\begin{BoxDefinition}[标量三重积]
    矢量$\vb*{A},\vb*{B},\vb*{C}$的\uwave{标量三重积}(Scalar Triple Product)被定义为
    \begin{Equation}
        \vb*{A}\cdot(\vb*{B}\times\vb*{C})
    \end{Equation}
\end{BoxDefinition}
\begin{BoxDefinition}[矢量三重积]
    矢量$\vb*{A},\vb*{B},\vb*{C}$的\uwave{矢量三重积}(Vector Triple Product)被定义为
    \begin{Equation}
        \vb*{A}\times(\vb*{B}\times\vb*{C})
    \end{Equation}
\end{BoxDefinition}

标量三重积比较重要的是下面的行列式表述
\begin{BoxFormula}[标量三重积的行列式表示]
    标量三重积可以表述为
    \begin{Equation}
        \vb*{A}\cdot\qty(\vb*{B}\times\vb*{C})=
        \begin{vmatrix}
            \vb*{A}\\
            \vb*{B}\\
            \vb*{C}\\
        \end{vmatrix}=
        \begin{vmatrix}
            A_x&A_y&A_z\\
            B_x&B_y&B_z\\
            C_x&C_y&C_z
        \end{vmatrix}
    \end{Equation}
\end{BoxFormula}

标量三重积具有很清楚的数学意义,它代表$\vb*{A},\vb*{B},\vb*{C}$围城的平行六面体的提及。

标量三重积具有轮换对称性,这就是说$\vb*{A},\vb*{B},\vb*{C}$和$\vb*{B},\vb*{C},\vb*{A}$和$\vb*{C},\vb*{A},\vb*{B}$的标量三重积其实是相同的,因为根据线性代数的知识,交换行列式的任意两行将会产生一个负号,而上述两两之间(对应行列式)的变换,均需要作两次行交换,产生的负号抵消,因而是相等的,即
\begin{BoxFormula}[标量三重积的轮换对称性]
    标量三重积具有轮换对称性
    \begin{Equation}
        \vb*{A}\cdot(\vb*{B}\times\vb*{C})=
        \vb*{B}\cdot(\vb*{C}\times\vb*{A})=
        \vb*{C}\cdot(\vb*{B}\times\vb*{A})
    \end{Equation}
\end{BoxFormula}
矢量三重积比较重要的,是下面的展开式
\begin{BoxFormula}[拉格朗日公式]
    矢量三重积具有以下展开式
    \begin{Equation}
        \vb*{A}\times(\vb*{B}\times \vb*{C})=
        \vb*{B}(\vb*{A}\cdot\vb*{C})-
        \vb*{C}(\vb*{A}\cdot\vb*{B})
    \end{Equation}
\end{BoxFormula}
矢量三重积的展开公式可以“后面的出租车back cab”形象记忆,即BAC(K) CAB。

矢量三重积的一种变式是
\begin{Gather}[6pt]
    \vb*{A}\times(\vb*{B}\times \vb*{C})=
    \vb*{B}(\vb*{A}\cdot\vb*{C})-
    \vb*{C}(\vb*{A}\cdot\vb*{B})\\
    (\vb*{A}\times \vb*{B})\times\vb*{C}=
    \vb*{B}(\vb*{C}\cdot\vb*{A})-
    \vb*{A}(\vb*{C}\cdot\vb*{B})
\end{Gather}
如何同时记忆这两者呢,以下是一些诀窍
\begin{enumerate}
    \item 结果总是括号里那两个矢量的线性组合。
    \item 结果的线性组合中,原先位于中间的矢量(即$\vb*{B}$)的系数为正,另一矢量的系数为负。
    \item 矢量前的系数由另外两个矢量的点积构成,点积满足交换律,故这里的顺序不重要。
\end{enumerate}
矢量三重积的展开公式的一个直接结果是下面的恒等式
\begin{BoxFormula}[雅可比恒等式]
    矢量三重积满足以下恒等式
    \begin{Equation}
        \vb*{A}\times(\vb*{B}\times \vb*{C})+\vb*{B}\times(\vb*{C}\times \vb*{A})+\vb*{C}\times(\vb*{A}\times \vb*{B})=\vb*{0}
    \end{Equation}
\end{BoxFormula}

