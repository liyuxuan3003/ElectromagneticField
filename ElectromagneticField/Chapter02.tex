\chapter{电磁场的基本定律}
电磁学的三大基本定律(库伦定律、安培力定律、法拉第电磁感应定律)的提出,标志着人类对宏观电磁现象的认识从定性阶段到定量阶段的飞跃。以三大实验定律为基础,麦克斯韦提出了两个基本假设(涡旋电场假设、位移电流假设),进而归纳出总结出描述宏观电磁现象的基本定律,麦克斯韦方程组。麦克斯韦方程组的建立,是对电磁理论进步的重大贡献,以麦克斯韦方程组为核心的宏观电磁理论,是研究电磁现象和解决现代工程电磁问题的理论基础。

在本章中,首席将介绍产生电磁场的源,即电荷和电流,再从实验定律引入描述电磁场的基本物理量,即电场强度$\vb*{E}$和磁感强度$\vb*{B}$,并讨论其散度特性和旋度特性。接着讨论介质的电磁特性,然后讨论涡旋电场和位移电流,最后,引出麦克斯韦方程组并讨论电磁场的边界条件。

\section{电荷守恒与电流连续性}
电荷周围要产生电场,电流周围要产生磁场,电荷电流是产生电磁场的源量。

\subsection{电荷与电荷密度}
自然界中存在两种\uwave{电荷}(Electric Charge):\uwave{正电荷}和\uwave{负电荷}。带电体所带电量称为\uwave{电荷量},电荷量并非是连续变化的,电荷量只能取值基本电荷$e=1.602\times 10^{-19}\si{C}$的整数倍,即,电荷是以离散方式分布的。基本电荷$e$恰好就是质子和电子的电荷量,质子带电$e$,电子带电$-e$。

电荷通常以$q$表示,尽管我们说,电荷实际上是离散分布的,但在研究宏观电磁现象时,我们所观察到的是带电体上大量微观带电粒子的总体效应,而带电粒子的线度又远小于带电体的线度。因此,我们可以近似认为,电荷是连续分布在带电体上的,并可以用电荷密度来描述。

\begin{BoxDefinition}[电荷体密度]*
    连续分布在体积$V$内的电荷,称为\uwave{体分布电荷}或\uwave{体电荷}(Volume Charge)。

    体电荷的分布,用\uwave{电荷体密度}(Volume Charge Density)表示
    \begin{Equation}
        \rho(\vb*{r},t)=\Lim[\delt{V}\to 0]\frac{\delt{q}(\vb*{r},t)}{\delt{V}}=\dv{q(\vb*{r},t)}{V}
    \end{Equation}
    体积$V$内的电荷量可以用电荷体密度$\rho(\vb*{r},t)$在$V$内的积分求出
    \begin{Equation}
        q(t)=\Itnt[V]\rho(\vb*{r},t)\dd{V}
    \end{Equation}
\end{BoxDefinition}

除了电荷体密度,亦有适用于面分布的面密度和线分布的线密度。

\begin{BoxDefinition}[电荷面密度]
    连续分布在曲面$S$内的电荷,称为\uwave{面分布电荷}或\uwave{面电荷}(Surface Charge)。

    面电荷的分布,用\uwave{电荷面密度}(Surface Charge Density)表示
    \begin{Equation}
        \rho_S(\vb*{r},t)=\Lim[\delt{S}\to 0]\frac{\delt{q}(\vb*{r},t)}{\delt{S}}=\dv{q(\vb*{r},t)}{S}
    \end{Equation}
    曲面$S$内的电荷量可以用电荷面密度$\rho_S(\vb*{r},t)$在$V$内的积分求出
    \begin{Equation}
        q(t)=\Isnt[S]\rho_S(\vb*{r},t)\dd{S}
    \end{Equation}
\end{BoxDefinition}

\begin{BoxDefinition}[电荷线密度]
    连续分布在曲线$l$内的电荷,称为\uwave{线分布电荷}或\uwave{线电荷}(Linear Charge)。

    线电荷的分布,用\uwave{电荷线密度}(Linear Charge Density)表示
    \begin{Equation}
        \rho_l(\vb*{r},t)=\Lim[\delt{l}\to 0]\frac{\delt{q}(\vb*{r},t)}{\delt{l}}=\dv{q(\vb*{r},t)}{l}
    \end{Equation}
    曲线$l$内的电荷量可以用电荷线密度$\rho_l(\vb*{r},t)$在$l$内的积分求出
    \begin{Equation}
        q(t)=\Int[l]\rho_l(\vb*{r},t)\dd{l}
    \end{Equation}
\end{BoxDefinition}

当带电体的线度,远小于观察点至带电体的距离时,此时,带电体的形状和电荷分布以及无关紧要,带电体可以近似为一个电荷集中在中心的几何点模型,称为\uwave{点电荷}(Point Charge)。

点电荷的概念在电磁理论中有很重要的地位,就像质点与之经典力学那般。

点电荷是电荷分布的一种极限情况,点电荷可以视为一个体积很小而带电量一定的带电球体在半径$a\to 0$时的极限。设电荷$q(t)$分布在位于$\vb*{r}'$半径为$a$的球体内,那么
\begin{itemize}
    \item 在$|\vb*{r}-\vb*{r}'|>a$的球外区域,电荷密度显然为零。
    \item 在$|\vb*{r}-\vb*{r}'|<a$的球内区域,电荷密度将为非常大的数值(由于球体很小),当$a\to 0$时球心$\vb*{r}=\vb*{r'}$处的电荷密度将趋于无穷大,而$\vb*{r}\neq\vb*{r}'$处的电荷密度则均为零,但整个极限过程中,电荷总量保持为$q(t)$不变。这也就是说,电荷密度$\rho(\vb*{r},t)$在全空间的积分是定值,在$\vb*{r}=\vb*{r}'$为无穷大,在$\vb*{r}\neq\vb*{r}'$为零,这在数学上完全符合狄拉克函数的特性。
\end{itemize}
因此,点电荷的电荷密度可以用狄拉克函数描述。
\begin{BoxDefinition}[点电荷]
    点电荷的电荷密度可以用狄拉克函数描述
    \begin{Equation}
        \rho(\vb*{r},t)=q(t)\dirac(\vb*{r}-\vb*{r}')
    \end{Equation}
    其中,$q(t)$是点电荷的电荷量,$\vb*{r}'$是点电荷的位矢。
\end{BoxDefinition}

\subsection{电流与电流密度}\vspace{-0.25cm}
\begin{BoxDefinition}[电流]
    \uwave{电流}(Electric Current)是由电荷运动形成的,定义为
    \begin{Equation}
        i(t)=\Lim[\delt{t}\to 0]\frac{\delt{q}(t)}{\delt{t}}=\dv{q(t)}{t}
    \end{Equation}
    其中,$\delt{q}(t)$表示在$\delt{t}$的时间内通过某一曲面$S$的电荷量,$i(t)$即通过曲面$S$的电流。
\end{BoxDefinition}

电流的意义是单位时间通过某一曲面$S$的电荷量,因此,我们不能说某一点的电流是多少,但是,我们又迫切的需要了解电流的分布情况,介于不同点处的电荷运动方向往往是不同的。

电流密度$\vb*{J}(\vb*{r},t)$就是为此而引入的,在$\vb*{r}$处的电流密度是如何计算的呢?我们在$\vb*{r}$处垂直电荷运动的方向上取一个面积元$\delt{S}$,其法向单位矢量为$\vb*{e}_\text{n}$,电流密度的大小即该面积元上的电流$\delt{i}$与该面积元$\delt{S}$的比在$\delt{S}\to 0$的极限,电流密度的方向就是该处$\vb*{e}_\text{n}$的方向。

\begin{BoxDefinition}[电流密度]
    \uwave{电流密度}(Current Density)描述了电流的分布
    \begin{Equation}
        \vb*{J}(\vb*{r},t)=\vb*{e}_\text{n}\Lim[\delt{S}\to 0]\frac{\delt{i}(\vb*{r})}{\delt{S}}=\vb*{e}_\text{n}\dv{i(\vb*{r})}{S}
    \end{Equation}
    电流则是电流密度的通量,通过曲面$S$的电流可以被表示为
    \begin{Equation}
        i(t)=\Isnt[S]\vb*{J}(\vb*{r},t)\cdot\dd{\vb*{S}}
    \end{Equation}
\end{BoxDefinition}
\begin{BoxFormula}[电流密度与电荷密度]
    电流密度,是电荷密度与电荷运动速度的乘积
    \begin{Equation}
        \vb*{J}(\vb*{r},t)=\rho\vb*{v}
    \end{Equation}
\end{BoxFormula}\nopagebreak

上述的电流体密度$\vb*{J}(\vb*{r},t)$描述的是体分布电流,我们还可以定义电流面密度$\vb*{J}_S(\vb*{r},t)$来描述面分布电流。但是,我们没有“电流线密度”的概念,因为“电流线密度”就是“电流$i$”,这种线电流的模型是很常用的,例如导线的横截面相较其长度而言很小,就可以视为线电流。\goodbreak

在分析电磁场时,常会使用\uwave{电流元}(Current Element)的概念,对于线电流,我们会沿电流流动方向取一个线元矢量$\dd{\vb*{l}}$,并将$i\dd{\vb*{l}}$称为电流元。而对于体分布电流和面分布电流,其电流元分别是$\vb*{J}\dd{V}$和$\vb*{J}_S\dd{S}$。在某种意义上,载流导体可以视为由电流元“积分”而来的。


\subsection{电荷守恒定律与电流连续性方程}
电荷是守恒的,它既不能被创造,也不能被消灭,只能从物体的一个部分转移到另一部分,或者从一个物体转移到另一个物体,这就是所谓\uwave{电荷守恒定律}(Law of Charge Conservation)。

电荷守恒定律的直接结果,就是\uwave{电流连续性方程}(Continuity Equation)。

\begin{BoxEquation}[电流连续性方程]
    电流连续性方程是指,电流场$\vb*{J}(\vb*{r},t)$满足以下方程
    \begin{Equation}
        \div\vb*{J}+\pdv{\rho}{t}=0
    \end{Equation}
    特别的,对于恒定电流场
    \begin{Equation}
        \div{\vb*{J}}=0
    \end{Equation}
\end{BoxEquation}
\begin{Proof}
    任取一个边界曲面为$S$的空间区域$V$,应有
    \begin{Equation}&[1]
        \Isot[S]{\vb*{J}\cdot\dd{\vb*{S}}}=-\dv{q}{t}=-\dv{t}\Itnt[V]\rho\dd{V}
    \end{Equation}
    这是因为,\xrefpeq{1}左端代表的是单位时间内从$S$流出的电荷,\xrefpeq{1}右端代表的是单位时间在空间区域$V$内减少的电荷,根据前述的电荷守恒定律,这两者很明显应该是相等的。

    由于$V$不会随时间变化,不妨将\xrefpeq{1}右端对时间的求导置于积分内
    \begin{Equation}&[2]
        \Isot[S]\vb*{J}\cdot\dd{\vb*{S}}=-\Itnt[V]\pdv{\rho}{t}\dd{V}
    \end{Equation}
    就\xrefpeq{2}的左端运用\fancyref{thm:散度定理}
    \begin{Equation}
        \Itnt[V]\div\vb*{J}\dd{V}=-\Itnt[V]\pdv{\rho}{t}\dd{V}
    \end{Equation}
    即
    \begin{Equation}
        \Itnt[V]\qty(\div\vb*{J}+\pdv{\rho}{t})\dd{V}=0
    \end{Equation}
    由于$V$是任取的
    \begin{Equation}
        \div\vb*{J}+\pdv{\rho}{t}=0
    \end{Equation}
    而恒定电流场中,电荷分布$\rho$不随时间变化,因此$\pdv*{\rho}{t}=0$。
\end{Proof}
电流连续性方程指出,\empx{时变电流场是有散场}
\begin{itemize}
    \item 电流场在电荷密度随时间减少的地方,发出电流线,构成电流场的源。
    \item 电流场在电荷密度随时间增加的地方,终止电流线,构成电流场的汇。
\end{itemize}
恒定电流场的电荷密度不随时间变化,因此没有源或汇,换言之,\empx{恒定电流场是无散场}。


\section{真空中静电场的基本规律}

电荷周围的空间存在电场,电场对电荷会产生作用力,称为电场力,这就是电场的基本特征。

\subsection{电场的基本特征}
\begin{BoxLaw}[电场的基本特性]
    若电场中存在电荷$q$,则其所受到的电场力,具有以下特性
    \begin{Equation}
        \vb*{F}_\text{c}=q\vb*{E}
    \end{Equation}
    其中$\vb*{E}$称为\uwave{电场强度}(Electric Field Intensity),单位是$\si{V\cdot m^{-1}}$。
\end{BoxLaw}

\subsection{电场的基本实验定律}
在1785年,法国科学家库伦(Coulomb)通过著名的“扭秤实验”,总结出真空中两个静止电荷$q_1$和$q_2$间的相互作用力的规律,称为\uwave{库伦定律}(Coulomb's Law),这是电场的基本实验定律,因此电场力也被称为\uwave{库伦力}(Coulomb Force)。库伦定律的内容可以叙述如下
\begin{BoxLaw}[库伦定律]
    设点电荷$q_1$和$q_2$的位矢分别为$\vb*{r}_1$和$\vb*{r}_2$,记$\vb*{R}_{12}=\vb*{r}_2-\vb*{r}_1$是$q_1$指向$q_2$的矢量。

    那么,$q_1$在$q_2$上产生的作用力$\vb*{F}_{12}$为
    \begin{Equation}
        \vb*{F}_{12}=\frac{q_1q_2}{4\pi\varepsilon_0R_{12}^3}\vb*{R}_{12}
    \end{Equation}
    其中物理常数$\varepsilon_0$被称为\uwave{真空电容率}(Vacuum Permittivity)\footnote[2]{除了真空电容率,有时$\varepsilon_0$也被称为真空介电常数。},其值为
    \begin{Equation}
        \varepsilon_0=(1/36\pi)\times 10^{-9}\si{F\cdot m^{-1}}
    \end{Equation}
\end{BoxLaw}

库伦定律指出,电荷间的作用满足平方反比关系,同号电荷相互排斥,异号电荷相互吸引。

库伦定律讨论的是两个电荷间的相互作用,设真空中有$N$个点电荷$q_1,q_2,\cdots,q_N$分别位于位矢$\vb*{r}_1',\vb*{r}_2',\cdots,\vb*{r}_N'$处,它们将作为电场的源点。那么,位于场点$\vb*{r}$处的点电荷$q$所受到的库伦力$\vb*{F}$,就可以依照力的叠加原理,视为$q,q_1$、$q,q_2$、$\cdots$、$q,q_N$两两之间的库伦力的和
\begin{Equation}
    \vb*{F}=\Sum[i=1][N]\frac{qq_iR_i}{4\pi\varepsilon_0R_i^3}=\frac{q}{4\pi\varepsilon_0}\Sum[i=1][N]\frac{q_i\vb*{R}_i}{R_i^3}
\end{Equation}
其中$\vb*{R}_i=\vb*{r}-\vb*{r}_i$,即由场点指向各个源点的矢量。

而根据\fancyref{law:电场的基本特性},我们很容易得出测试点电荷$q$处的电场强度应为
\begin{Equation}
    \vb*{E}(\vb*{r})=\frac{1}{4\pi\varepsilon_0}\Sum[i=1][N]\frac{q_i\vb*{R}_i}{R_i^3}
\end{Equation}
而对于连续带电体,上式将转化为积分
\begin{BoxFormula}[电场强度]
    电场强度$\vb*{E}(\vb*{r})$符合以下规律
    \begin{Equation}
        \vb*{E}(\vb*{r})=\frac{1}{4\pi\varepsilon_0}\Itnt[V]\frac{\rho(\vb*{r}')\vb*{R}}{R^3}\dd{V'}
    \end{Equation}
    其中$\vb*{R}=\vb*{r}-\vb*{r}'$,而$\rho(\vb*{r}')$则给出$V$中源点$\vb*{r}'$处的电荷密度。
\end{BoxFormula}

\subsection{距离反比的拉普拉斯}
在开始讨论静电场的性质之前,我们先需要补充一个重要的数学公式。\cite{W2}
\begin{BoxFormula}[距离反比的拉普拉斯]
    设位矢$\vb*{r}$,$r=\abs{\vb*{r}}$,则有
    \begin{Equation}
        \laplacian(\frac{1}{r})=-4\pi\dirac(\vb*{r})
    \end{Equation}
    若记$\vb*{R}=\vb*{r}-\vb*{r}'$,$\vb*{r}$为场点,$\vb*{r}'$为源点常量,则有
    \begin{Equation}
        \laplacian(\frac{1}{R})=-4\pi\dirac(\vb*{R})=-4\pi\dirac(\vb*{r}-\vb*{r}')
    \end{Equation}
\end{BoxFormula}
\begin{Proof}
    根据\fancyref{fml:球坐标系的拉普拉斯},若$r\neq 0$
    \begin{Equation}
        \laplacian(\frac{1}{r})=\frac{1}{r^2}\pdv{r}\qty[r^2\pdv{(1/r)}{r}]=\frac{1}{r^2}\pdv{r}\qty[r^2\qty(-\frac{1}{r^2})]=0
    \end{Equation}
    而若$r\neq 0$,上式不再适用,我们试着对$\laplacian(1/r)$在$V$上积分
    \begin{Equation}
        \qquad\qquad\qquad
        I=\Itnt[V]\laplacian(\frac{1}{r})\dd{V}=\Itnt[V_0]\laplacian(\frac{1}{r})\dd{V}+\Itnt[V-V_{0}]\laplacian(\frac{1}{r})\dd{V}
        \qquad\qquad\qquad
    \end{Equation}
    这里将$V$拆分为两部分,$V_0$是以原点为球心,半径为$r_0$的球形区域,半径$r_0$可以任取以确保$V_0$完全在$V$之内。而$V-V_0$是剩余区域,由于$V-V_0$不包含原点,因此其积分为零。

    因此,在$V$上的积分就可以转化为球形区域$V_0$上的积分
    \begin{Equation}
        I=\Itnt[V]\laplacian(\frac{1}{r})\dd{V}=
        \Itnt[V_0]\laplacian(\frac{1}{r})\dd{V}
    \end{Equation}
    运用\fancyref{thm:散度定理}
    \begin{Equation}
        I=
        \Itnt[V_0]\div\grad(\frac{1}{r})\dd{V}=
        \Isot[S_0]\grad(\frac{1}{r})\cdot\dd{\vb*{S}}
    \end{Equation}
    其中$S_0$为$V_0$的边界曲面,即半径为$r_0$的球面,而球面上的积分是可以计算的
    \begin{Equation}
        I=\Isot[S_0]\grad(\frac{1}{r})\cdot\frac{\vb*{r}_0}{r_0}\dd{S}=\Int[0][2\pi]\Int[0][\pi]\grad(\frac{1}{r})\cdot\frac{\vb*{r}_0}{r_0}r_0^2\sin\theta\dd{\theta}\dd\phi
    \end{Equation}
    代入\fancyref{fml:距离反比的梯度},注意到$r_0$被完全约去
    \begin{Equation}
        \qquad\qquad
        I=\Int[0][2\pi]\Int[0][\pi]-\frac{\vb*{r}_0}{r_0^3}\cdot\frac{\vb*{r}_0}{r_0}r_0^2\sin\theta\dd{\theta}\dd\phi=\Int[0][2\pi]\Int[0][\pi]-\sin\theta\dd{\theta}\dd\phi=-4\pi
        \qquad\qquad
    \end{Equation}
    在这里,我们看到,$\laplacian(1/r)$在$r\neq 0$时处处为零,但是,$\laplacian(1/r)$在包含$r=0$的空间上的积分却并不是零,而是有限值$-4\pi$,这就意味着$\laplacian(1/r)$需要用狄拉克函数表示,即
    \begin{Equation}*
        \laplacian(\frac{1}{r})=-4\pi\dirac(\vb*{r})\qedhere
    \end{Equation}
\end{Proof}\vspace{-0.25cm}

\subsection{静电场的散度}
\begin{BoxProperty}[静电场的散度]*
    静电场的散度满足
    \begin{Equation}
        \div\vb*{E}=\frac{\rho}{\varepsilon_0}
    \end{Equation}
    该性质也可以改写为积分形式
    \begin{Equation}
        \Isot[S]\vb*{E}\cdot\dd{\vb*{S}}=\frac{1}{\varepsilon_0}\Itnt[V]\rho\dd{V}=\frac{q}{\varepsilon_0}
    \end{Equation}
    该结论称为\uwave{高斯定律}(Gauss's Law)。

    该式表明,电场强度$\vb*{E}$在闭曲面上的通量等于闭曲面内的总电荷与$\varepsilon_0$之比。
\end{BoxProperty}

\begin{Proof}
    根据\fancyref{fml:电场强度}
    \begin{Equation}&[1]
        \vb*{E}(\vb*{r})=\frac{1}{4\pi\varepsilon_0}\Itnt[V]\frac{\rho(\vb*{r}')\vb*{R}}{R^3}\dd{V'}
    \end{Equation}\goodbreak
    根据\fancyref{fml:距离反比的梯度},$\grad(1/R)=-\vb*{R}/R^3$
    \begin{Equation}&[2]
        \vb*{E}(\vb*{r})=-\frac{1}{4\pi\varepsilon_0}\Itnt[V]\rho(\vb*{r}')\grad(\frac{1}{R})\dd{V'}
    \end{Equation}
    在\xrefpeq{2}两端取散度
    \begin{Equation}&[3]
        \div\vb*{E}(\vb*{r})=\div\qty[-\frac{1}{4\pi\varepsilon_0}\Itnt[V]\rho(\vb*{r}')\grad(\frac{1}{R})\dd{V'}]
    \end{Equation}
    由于微分算符$\grad$是对场点$\vb*{r}$的坐标的微分,而\xrefpeq{3}的积分是对源点$\vb*{r}'$坐标进行,故微分算符$\grad$可与积分运算交换顺序,而$\rho(\vb*{r}')$只与源点$\vb*{r}'$有关,也可以移到微分算符$\grad$之外
    \begin{Equation}
        \div\vb*{E}(\vb*{r})=-\frac{1}{4\pi\varepsilon_0}\Itnt[V]\rho(\vb*{r}')\div\grad(\frac{1}{R})\dd{V'}
    \end{Equation}
    梯度的散度即拉普拉斯算符
    \begin{Equation}
        \div\vb*{E}(\vb*{r})=-\frac{1}{4\pi\varepsilon_0}\Itnt[V]\rho(\vb*{r}')\laplacian(\frac{1}{R})\dd{V'}
    \end{Equation}
    利用\fancyref{fml:距离反比的拉普拉斯},$\laplacian(1/R)=-4\pi\dirac(\vb*{R})=-4\pi\dirac(\vb*{r}-\vb*{r}')$
    \begin{Equation}
        \div\vb*{E}(\vb*{r})=\frac{1}{\varepsilon_0}\Itnt[V]\rho(\vb*{r}')\dirac(\vb*{r}-\vb*{r}')\dd{V'}
    \end{Equation}
    利用狄拉克函数的筛选性质
    \begin{Equation}*
        \div\vb*{E}=\frac{\rho}{\varepsilon_0}\qedhere
    \end{Equation}
\end{Proof}

\subsection{静电场的旋度}
\begin{BoxProperty}[静电场的旋度]
    静电场的旋度满足
    \begin{Equation}
        \curl\vb*{E}=\vb*{0}
    \end{Equation}
    该性质也可以改写为积分形式
    \begin{Equation}
        \Ilot[C]\vb*{E}\cdot\dd{\vb*{l}}=0
    \end{Equation}
    该式表明,电场强度$\vb*{E}$在闭曲线上的环流等于零,即,静电场是无旋场。
\end{BoxProperty}
\begin{Proof}
    我们回到静电场散度中的\xrefpeq[静电场的散度]{2}
    \begin{Equation}&[1]
        \vb*{E}(\vb*{r})=-\frac{1}{4\pi\varepsilon_0}\Itnt[V]\rho(\vb*{r}')\grad(\frac{1}{R})\dd{V'}
    \end{Equation}
    由于积分是关于源点$\vb*{r}'$,而微分算符关于常点$\vb*{r}$,故微分算符可以提至积分外
    \begin{Equation}&[2]
        \vb*{E}(\vb*{r})=-\grad[\frac{1}{4\pi\varepsilon_0}\Itnt[V]\frac{\rho(\vb*{r'})}{R}\dd{V'}]
    \end{Equation}
    在\xrefpeq{2}两端取旋度
    \begin{Equation}&[3]
        \curl\vb*{E}=-\curl\grad[\frac{1}{4\pi\varepsilon_0}\Itnt[V]\frac{\rho(\vb*{r'})}{R}\dd{V'}]
    \end{Equation}
    在\xrefpeq{3}右端,是一个标量场梯度的旋度,而根据\fancyref{ppt:标量场的梯度无旋}
    \begin{Equation}*
        \curl\vb*{E}=\vb*{0}\qedhere
    \end{Equation}
\end{Proof}