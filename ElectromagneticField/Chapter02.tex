\chapter{电磁场的基本定律}
电磁学的三大基本定律(库伦定律、安培力定律、法拉第电磁感应定律)的提出,标志着人类对宏观电磁现象的认识从定性阶段到定量阶段的飞跃。以三大实验定律为基础,麦克斯韦提出了两个基本假设(涡旋电场假设、位移电流假设),进而归纳出总结出描述宏观电磁现象的基本定律,麦克斯韦方程组。麦克斯韦方程组的建立,是对电磁理论进步的重大贡献,以麦克斯韦方程组为核心的宏观电磁理论,是研究电磁现象和解决现代工程电磁问题的理论基础。

在本章中,首席将介绍产生电磁场的源,即电荷和电流,再从实验定律引入描述电磁场的基本物理量,即电场强度$\vb*{E}$和磁感强度$\vb*{B}$,并讨论其散度特性和旋度特性。接着讨论介质的电磁特性,然后讨论涡旋电场和位移电流,最后,引出麦克斯韦方程组并讨论电磁场的边界条件。