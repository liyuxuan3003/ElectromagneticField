\section{格林定理}
\uwave{格林定理}又称为\uwave{格林恒等式},是由散度定理导出的重要恒等式,它描述了两个标量场之间满足的关系,如果已知一个标量场,就可以用格林定理求出另外一个标量场,这是很有用的。

格林恒等式共有两条,在微积分中我们都已经证明过了,这里直接给出结论。
\begin{BoxFormula}[格林第一恒等式]
    \uwave{格林第一恒等式}(Green's First Identity)是指
    \begin{Equation}
        \Itnt[V](u\laplacian v+\grad u\cdot\grad v)\dd{V}=\Isot[S]u\grad v\cdot\dd\vb*{S}
    \end{Equation}
\end{BoxFormula}
\begin{BoxFormula}[格林第二恒等式]
    格林第二恒等式(Green's Second Identity)是指
    \begin{Equation}
        \Itnt[V](u\laplacian v-v\laplacian u)\dd{V}=\Isot[S](u\grad v-v\grad u)\cdot\dd\vb*{S}
    \end{Equation}
\end{BoxFormula}

格林第二恒等式,其实就是将第一恒等式减去$u,v$交换的第一恒等式。