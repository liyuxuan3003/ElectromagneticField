\section{格林定理}
格林定理又称为格林恒等式,是由散度定理导出的重要恒等式,它描述了两个标量场之间满足的关系,如果已知一个标量场,就可以用格林定理求出另外一个标量场,这是很有用的。

格林恒等式共有两条,在微积分中我们都已经证明过了,这里直接给出结论。
\begin{BoxFormula}[格林第一恒等式]
    格林第一恒等式是指
    \begin{Equation}
        \Itnt[V](u\laplacian v+\grad u\cdot\grad v)\dd{V}=\Isot[S]u\grad v\cdot\dd\vb*{S}
    \end{Equation}
\end{BoxFormula}
\begin{BoxFormula}[格林第二恒等式]
    格林第二恒等式是指
    \begin{Equation}
        \Itnt[V](u\laplacian v-v\laplacian u)\dd{V}=\Isot[S](u\grad v-v\grad u)\cdot\dd\vb*{S}
    \end{Equation}
\end{BoxFormula}