\section{电磁场的波动方程}
电磁场总共有$\vb*{H},\vb*{E},\vb*{B},\vb*{D}$四个矢量,两个有关电场,两个有关磁场,但在线性各向同性的介质中,运用$\vb*{D}=\varepsilon\vb*{E}$和$\vb*{B}=\mu\vb*{H}$的本构关系,可以只保留一个电场矢量和一个磁场矢量。通常来说,讨论电磁场的波动方程时,我们会选择以电场强度$\vb*{E}$和磁场强度$\vb*{H}$作为自变量。

\begin{BoxEquation}[电磁场的波动方程]
    电磁场中,矢量$\vb*{E},\vb*{H}$满足以下波动方程
    \begin{Gather}[10pt]
        \laplacian\vb*{E}-\mu\varepsilon\pdv[2]{\vb*{E}}{t}=\mu\pdv{\vb*{J}}{t}+\varepsilon^{-1}\grad\rho\xlabelpeq{A}\\
        \laplacian\vb*{H}-\mu\varepsilon\pdv[2]{\vb*{H}}{t}=-\curl\vb*{J}\xlabelpeq{B}
    \end{Gather}
\end{BoxEquation}
\begin{Proof}
    根据\fancyref{eqt:麦克斯韦方程组},应用本构关系
    \begin{Gather}[8pt]
        \curl\vb*{H}=\vb*{J}+\varepsilon\pdv{\vb*{E}}{t}\xlabelpeq{1}\\
        \curl\vb*{E}=-\mu\pdv{\vb*{H}}{t}\xlabelpeq{2}\\
        \div\vb*{H}=0\xlabelpeq{3}\\
        \div\vb*{E}=\frac{\rho}{\varepsilon}\xlabelpeq{4}
    \end{Gather}
    现在$\vb*{H},\vb*{E}$是相互耦合的,我们的目标就是将两者分离为两个独立的方程。

    \paragraph{电场强度部分}
    在\xrefpeq{2}两端取旋度,得到
    \begin{Equation}&[5]
        \curl(\curl\vb*{E})=-\mu\pdv{t}\qty(\curl\vb*{H})
    \end{Equation}
    在\xrefpeq{5}中就$\curl\vb*{H}$代入\xrefpeq{1}
    \begin{Equation}&[6]
        \curl(\curl\vb*{E})=-\mu\pdv{t}\qty(\vb*{J}+\varepsilon\pdv{\vb*{E}}{t})=-\mu\pdv{\vb*{J}}{t}-\mu\varepsilon\pdv[2]{\vb*{E}}{t}
    \end{Equation}
    而另外一方面,根据\fancyref{fml:矢量拉普拉斯的转化}和\xrefpeq{4}
    \begin{Equation}&[7]
        \curl(\curl\vb*{E})=\grad(\div\vb*{E})-\laplacian\vb*{E}=\varepsilon^{-1}\grad\rho-\laplacian\vb*{E}
    \end{Equation}
    现在联立\xrefpeq{6}和\xrefpeq{7}
    \begin{Equation}&[8]
        \laplacian\vb*{E}-\mu\varepsilon\pdv[2]{\vb*{E}}{t}=\mu\pdv{\vb*{J}}{t}+\varepsilon^{-1}\grad\rho
    \end{Equation}
    这就得到了\xrefpeq{A}。

    \paragraph{磁场强度部分}
    在\xrefpeq{1}两端取旋度,得到
    \begin{Equation}&[9]
        \curl(\curl\vb*{H})=\curl\vb*{J}+\varepsilon\pdv{t}(\curl\vb*{E})
    \end{Equation}
    在\xrefpeq{9}中就$\curl\vb*{E}$代入\xrefpeq{2}
    \begin{Equation}&[10]
        \curl(\curl\vb*{H})=\curl\vb*{J}-\mu\varepsilon\pdv[2]{\vb*{H}}{t}
    \end{Equation}
    而另外一方面,根据\fancyref{fml:矢量拉普拉斯的转化}和\xrefpeq{3}
    \begin{Equation}&[11]
        \curl(\curl\vb*{H})=\grad(\div\vb*{H})-\laplacian\vb*{H}=-\laplacian\vb*{H}
    \end{Equation}
    现在联立\xrefpeq{10}和\xrefpeq{11}
    \begin{Equation}
        \laplacian\vb*{H}-\mu\varepsilon\pdv[2]{\vb*{H}}{t}=-\curl\vb*{J}
    \end{Equation}
    这就得到了\xrefpeq{B}。
\end{Proof}
但是,矢量波动方程的求解是极为困难的,我们需要寻求更简单的方法。