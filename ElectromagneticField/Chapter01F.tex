\section{无散场的矢量位}

\subsection{无散场的定义}
\begin{BoxDefinition}[无散场的定义]
    若一个矢量场$\vb*{F}$的散度处处为零,即
    \begin{Equation}
        \div\vb*{F}=\vb*{0}
    \end{Equation}
    则称该矢量场为无散场,它完全由涡旋源产生。
\end{BoxDefinition}
根据\fancyref{thm:散度定理},无散场$\vb*{F}$在任意闭合曲面$S$的通量为$0$,即
\begin{Equation}
    \Isot[C]\vb*{F}\cdot\dd{\vb*{S}}=\Itnt[V]\div\vb*{F}\dd{V}=0
\end{Equation}
在这里,我们可以梳理一下,\empx{无旋场在闭曲线的环流为零},\empx{无散场在闭曲面的通量为零}。

\subsection{无散场的产生}
现在的问题是,什么样的场一定是无散场$\vb*{F}$呢?下面的结论将给我们答案。
\begin{BoxProperty}[矢量场的旋度无散]
    矢量场的旋度,必为无散场(旋度场无散)
    \begin{Equation}
        \div(\curl \vb*{A})=0
    \end{Equation}
\end{BoxProperty}\goodbreak
\begin{Proof}
    % 这个结论在数学物理方法中我们已经证明过了,但这里换一种更优雅的方式来完成。

    任取一个空间区域$V$,将$\curl(\grad u)$在$V$上积分
    \begin{Equation}&[1]
        I=\Itnt[V]\div(\curl\vb*{A})\cdot\dd{\vb*{S}}
    \end{Equation}
    设空间区域$V$的边界曲面为$S$,根据\fancyref{thm:散度定理}
    \begin{Equation}&[2]
        I=
        \Isot[S]\curl\vb*{A}\cdot\dd{\vb*{S}}
    \end{Equation}
    在闭合曲面$S$上作一条闭合曲线$C$,将$S$划分为$S_1$和$S_2$
    \begin{Equation}&[3]
        I=\Isnt[S_1]\curl\vb*{A}\cdot\dd{\vb*{S}}+\Isnt[S_2]\curl\vb*{A}\cdot\dd{\vb*{S}}
    \end{Equation}
    再次运用\fancyref{thm:旋度定理},注意到$S_1,S_2$的边界均为$C$,但取向相反
    \begin{Equation}
        I=\Ilot[C]\vb*{A}\cdot\dd{\vb*{l}}-\Ilot[C]\vb*{A}\cdot\dd{\vb*{l}}=0
    \end{Equation}
    将该积分结果代回\xrefpeq{1},考虑到区域$V$的任意性
    \begin{Equation}*
        \div(\curl\vb*{A})=0\qedhere
    \end{Equation}
\end{Proof}

\subsection{无散场的矢势}
既然矢量场的旋度为无散场,因此,若$\vb*{F}$是无散场,那么$\vb*{F}$必然可以表示为矢量场的旋度。
\begin{BoxDefinition}[无散场的矢势]
    若$\vb*{F}$是无散场,则必然存在矢量函数$\vb*{A}$使得
    \begin{Equation}
        \vb*{F}=\curl\vb*{A}
    \end{Equation}
    该矢量函数$\vb*{A}$就是无散场$\vb*{F}$的\uwave{矢量位}或\uwave{矢势}(Vector Potential)。
\end{BoxDefinition}


简而言之,\empx{无散场是其矢势的正旋度},注意,矢量位没有标量位定义中的负号。

在这里,我们有必要总结一下
\begin{itemize}
    \item 无旋场有标量位,无旋场是其标量位的负梯度,因为(标量的)梯度场无旋。
    \item 无散场有矢量位,无散场是其矢量位的负旋度,因为(矢量的)旋度场无散。
\end{itemize}

在电磁场中,我们往往会将对某个无散场性质的讨论,转化为对无散场的矢量位性质的讨论。\setpeq{矢量泊松方程}

无散场$\vb*{F}$的散度显然为零,这就是无散场的定义
\begin{Equation}&[1]
    \div\vb*{F}=\vb*{0}
\end{Equation}
无散场$\vb*{F}$的旋度则是有值的,设为$\vb*{G}$
\begin{Equation}&[2]
    \curl\vb*{F}=\vb*{G}
\end{Equation}
在上式中代入\fancyref{def:无散场的矢势}
\begin{Equation}&[3]
    \curl(\curl\vb*{A})=\vb*{G}
\end{Equation}
现在的问题是,这里遇到的“旋度的旋度”到底应该怎么求?\goodbreak

其实,我们只需要将$\grad$当作矢量,运用矢量三重积的\fancyref{fml:拉格朗日公式}
\begin{Equation}&[4]
    \curl(\curl\vb*{A})=\grad(\div\vb*{A})-(\div\grad)\vb*{A}
\end{Equation}
实际上,依照\fancyref{def:无散场的矢势}给出的无散场$\vb*{F}$的矢势$\vb*{A}$并不是唯一的,就像先前无旋场$\vb*{F}$的标势$u$间可以相差一个任意常数那样,矢势$\vb*{A}$的任意性是表现在它的散度可以是任意值,简洁起见,我们不妨假定其满足$\div\vb*{A}=0$的约束(之后会称之为库伦规范)。

将$\div\vb*{A}=0$的约束代入\xrefpeq{4}
\begin{Equation}&[5]
    (\div\grad)\vb*{A}=-\vb*{G}
\end{Equation}
这里$\div\grad$称为\uwave{矢量拉普拉斯算符},记为$\laplacian$,由此,我们就可以得到一个重要的事实。
\begin{BoxFormula}[矢量泊松方程]
    无散场$\vb*{F}$的旋度性质,可以等价的用无散场$\vb*{F}$的矢势的矢量泊松方程表示
    \begin{Equation}
        \laplacian\vb*{F}=-\vb*{G}
    \end{Equation}
    特别的,在$\vb*{G}=0$的部分,这就将给出矢量拉普拉斯方程
    \begin{Equation}
        \laplacian \vb*{F}=0
    \end{Equation}
    其中,$\vb*{G}$为$\vb*{F}$的旋度,$\vb*{A}$为$\vb*{F}$的矢势,即
    \begin{Equation}
        \div\vb*{F}=0\qquad \curl\vb*{F}=\vb*{G}\qquad \vb*{F}=\curl\vb*{A}
    \end{Equation}
\end{BoxFormula}

矢量拉普拉斯和标量拉普拉斯从数学形式上,都是$\div\grad$,即两个$\grad$的点积,从这一点看上两者是完全相同的,而不同之处在于,标量拉普拉斯可以理解为梯度的散度,但是,矢量拉普拉斯则不能这样理解,因为,矢量不能计算梯度(除非引入张量的概念,但这就太复杂了)。\goodbreak

矢量拉普拉斯计算时,往往会通过\xrefpeq{3}转化为常规运算
\begin{BoxFormula}[矢量拉普拉斯的转化]
    矢量拉普拉斯可以作以下转化
    \begin{Equation}
        \laplacian\vb*{A}=\grad(\div\vb*{A})-\curl(\curl\vb*{A})
    \end{Equation}
    即,矢量拉普拉斯,就是散度的梯度,减去旋度的旋度。
\end{BoxFormula}

矢量拉普拉斯在直角坐标下,可以视为矢量各个分离的标量拉普拉斯
\begin{BoxFormula}[矢量拉普拉斯的展开]
    矢量拉普拉斯在直角坐标系下,可以展开为
    \begin{Equation}
        \laplacian\vb*{A}=\vb*{e}_x\laplacian A_x+\vb*{e}_y\laplacian A_y+\vb*{e}_z\laplacian A_z
    \end{Equation}
\end{BoxFormula}
但注意,这仅对直角坐标系适用,在柱坐标系和球坐标系下该结论并不正确。

