\section{静磁场的能量}
静磁场和静电场一样,其能量都是定域在空间中的。
\begin{BoxFormula}[静磁场的能量]
    静磁场的能量定域在磁场不为零的空间中,能量密度满足
    \begin{Equation}
        w_\text{m}=\frac{1}{2}\vb*{B}\cdot\vb*{H}
    \end{Equation}
\end{BoxFormula}

\begin{Proof}
    设有$N$个回路$C_1,C_2,\cdots,C_N$且各回路$C_k$中的电流$i_k$从零开始以相同比例$\alpha\in[0,1]$上升,即$i_k=\alpha I_k$,很显然,在$\dd{t}$时间内,回路$C_k$中的电流将相应增加$\dd{i_k}=I_k\dd{\alpha}$。

    根据\fancyref{law:法拉第电磁感应定律},回路$C_k$中的感应电动势为
    \begin{Equation}&[1]
        \Emf_{\text{in}k}=-\dv{\Psi_k}{t}
    \end{Equation}
    而为了克服感应电动势,回路$C_k$的外加电压$u_k$应当满足
    \begin{Equation}&[2]
        u_k=-\Emf_{\text{in}k}=\dv{\Psi_k}{t}
    \end{Equation}
    电压与电荷的积即能量,因而在$\dd{t}$时间内回路$C_k$中增加的磁场能量$\dd{W}_{\text{m}k}$为
    \begin{Equation}&[3]
        \dd{W_{\text{m}k}}=u_k\dd{q_k}
    \end{Equation}
    在\xrefpeq{3}中代入\xrefpeq{2},并考虑到$\dd{q_k}=i_k\dd{t}$
    \begin{Equation}&[4]
        \dd{W_{\text{m}k}}=\dv{\Psi_k}{t}i_k\dd{t}
    \end{Equation}
    注意到$\dd{t}$被约去
    \begin{Equation}&[5]
        \dd{W_{\text{m}k}}=i_k\dd{\Psi_k}
    \end{Equation}
    我们将\xrefpeq{5}从$k=1$到$k=N$求和,即得$N$个回路在$\dd{t}$时间内增加的磁场能量
    \begin{Equation}&[6]
        \dd{W_\text{m}}=\Sum[k=1][N]\dd{W_{\text{m}k}}=\Sum[k=1][N]i_k\dd{\Psi_k}
    \end{Equation}
    现在的问题是,这里回路$C_k$上的磁链
    $\Psi_k$应当如何表示?事实上
    \begin{Equation}&[7]
        \Psi_k=\Sum[l=1][N]M_{kl}i_l
    \end{Equation}
    其中,$M_{kl}$是回路$C_k,C_l(l\neq k)$间的互感,$M_{kk}=L_{k}$是回路$C_k$的自感。

    将\xrefpeq{7}代入\xrefpeq{6}
    \begin{Equation}&[8]
        \dd{W_\text{m}}=\Sum[k=1][N]\Sum[l=1][N]i_kM_{kl}\dd{i_l}
    \end{Equation}
    根据证明首的论述,这里$i_k=\alpha I_k$,而$\dd{i_l}=I_l\dd{\alpha}$
    \begin{Equation}&[9]
        \dd{W_\text{m}}=\Sum[k=1][N]\Sum[l=1][N]M_{kl}I_kI_l\alpha\dd{\alpha}
    \end{Equation}
    将上式从$0$到$1$对$\alpha$积分
    \begin{Equation}&[10]
        W_\text{m}=\Sum[k=1][N]\Sum[l=1][N]M_{kl}I_kI_l\Int[0][1]\alpha\dd{\alpha}
    \end{Equation}
    即
    \begin{Equation}&[11]
        W_\text{m}=\frac{1}{2}\Sum[k=1][N]\Sum[l=1][N]M_{kl}I_kI_l
    \end{Equation}
    当充电完成后,\xrefpeq{7}即
    \begin{Equation}&[12]
        \Psi_k=\Sum[l=1][N]M_{kl}I_l
    \end{Equation}
    将\xrefpeq{12}代入\xrefpeq{11}
    \begin{Equation}&[13]
        W_\text{m}=\frac{1}{2}\Sum[k=1][N]I_k\Psi_k
    \end{Equation}
    而另外一方面,磁链也可以用磁矢势表述
    \begin{Equation}&[14]
        \Psi_k=\Isot[S_k]\vb*{B}\cdot\dd{\vb*{S}_k}=\Ilot[C_k]\curl\vb*{B}\cdot\dd{\vb*{l}_k}=\Ilot[C_k]\vb*{A}\cdot\dd{\vb*{l}_k}
    \end{Equation}
    将\xrefpeq{14}代入\xrefpeq{13}
    \begin{Equation}&[15]
        W_\text{m}=\frac{1}{2}\Sum[k=1][N]I_k\Ilot[C_k]\vb*{A}\cdot\dd{\vb*{l}_k}
    \end{Equation}
    而一般的说,电流总是体分布的,此时将$I_k\dd{\vb*{l}_k}$替换为$\vb*{J}\dd{V}$,求和替换为积分
    \begin{Equation}&[16]
        W_\text{m}=\frac{1}{2}\Itnt[V]\vb*{A}\cdot\vb*{J}\dd{V}
    \end{Equation}
    根据\fancyref{eqt:麦克斯韦方程组},代入$\vb*{J}=\curl\vb*{H}$
    \begin{Equation}&[17]
        W_\text{m}=\frac{1}{2}\Itnt[V]\vb*{A}\cdot(\curl\vb*{H})\dd{V}
    \end{Equation}
    这里需要运用矢量叉积的散度公式
    \begin{Equation}&[19]
        \div(\vb*{A}\times\vb*{H})=
        \vb*{H}\cdot(\curl\vb*{A})-
        \vb*{A}\cdot(\curl\vb*{H})
    \end{Equation}
    将\xrefpeq{19}代入\xrefpeq{17}
    \begin{Equation}&[20]
        W_\text{m}=\frac{1}{2}\Itnt[V]\vb*{H}\cdot(\curl\vb*{A})\dd{V}-\frac{1}{2}\Itnt[V]\div(\vb*{A}\times\vb*{H})\dd{V}
    \end{Equation}
    就\xrefpeq{20},第一项运用\fancyref{def:磁矢势},第二项运用\fancyref{thm:旋度定理}
    \begin{Equation}&[21]
        W_\text{m}=\frac{1}{2}\Itnt[V]\vb*{H}\cdot\vb*{B}\dd{V}-\frac{1}{2}\Isot[S]\vb*{A}\times\vb*{H}\cdot\dd{\vb*{S}}
    \end{Equation}
    由于当$V$趋于无限大时,我们有
    \begin{Equation}&[22]
        |\vb*{A}|\propto\frac{1}{r}\qquad
        |\vb*{H}|\propto\frac{1}{r^2}
    \end{Equation}
    即
    \begin{Equation}&[23]
        |\vb*{A}\times\vb*{H}|\propto\frac{1}{r^3}
    \end{Equation}
    因此,当$S$无限扩大,即$r\to\infty$时
    \begin{Equation}&[24]
        \frac{1}{2}\Isot[S]\vb*{A}\times\vb*{H}\cdot\dd{\vb*{S}}\propto\frac{1}{r^3}r^2\propto\frac{1}{r}\to 0
    \end{Equation}
    这样一来,\xrefpeq{21}就可以简化为
    \begin{Equation}
        W_\text{m}=\frac{1}{2}\Itnt[\R^3]\vb*{H}\cdot\vb*{B}\dd{V}
    \end{Equation}
    因此,能量密度就是
    \begin{Equation}*
        w_\text{m}=\frac{1}{2}\vb*{H}\cdot\vb*{B}\qedhere
    \end{Equation}
\end{Proof}