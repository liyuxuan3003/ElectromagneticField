\section{标量场的性质}
若物理量是一个标量,则该物理量所确定的场称为\uwave{标量场}(Scalar Field),常用$u$表示
\begin{Equation}
    u=u(\vb*{r},t)
\end{Equation}
而对于时不变的标量场,则可以表示为
\begin{Equation}
    u=u(\vb*{r})
\end{Equation}
在本节,我们就将研究标量场的性质。

\subsection{标量场的等值面}
标量场是一个关于三维空间点的函数,因此其可视化将成为一个麻烦的问题,介于我们没有第四个维度来表示场函数的值。常用的办法是,绘制一系列使$u(\vb*{r})$取值相同的空间曲面。
\begin{BoxDefinition}[标量场的等值面]
    标量场的可视化方法是为\uwave{等值面}(Isosurface),其是由任意常数$c$
    \begin{Equation}
        u(\vb*{r})=c
    \end{Equation}
    所确定的曲面族,称为\uwave{等值面族},该方程也称为\uwave{等值面方程}。
\end{BoxDefinition}
很明显,由于标量场函数$u(\vb*{r})$是单值的,对于某个$\vb*{r}$的取值,要么等于$c_1$,要么等于$c_2$,这就是说一个空间点$\vb*{r}$只可能存在于一个等值面上,换言之,\empx{标量场的等值面间是互不相交的}。

\subsection{标量场的方向导数和梯度}
标量场的场函数$u(\vb*{r})$仅描述了场量$u$的分布情况,而研究标量场的另外一个重要方面,就是研究场量$u$在场中任意一点,沿各个方向的变化规律,为此,引入方向导数和梯度的概念。

\begin{BoxDefinition}[标量场的方向导数]
    定义标量场$u(\vb*{r})$沿$\vb*{l}$的\uwave{方向导数}(Directional Derivative)
    \begin{Equation}
        \eval{\pdv{u}{l}}=\Lim[\delt{l}\to 0]\frac{u(\vb*{r}+\delt{\vb*{l}})-u(r)}{\delt{l}}
    \end{Equation}
\end{BoxDefinition}
标量场的方向导数仍然是一个标量场,需要注意的是,单说“方向导数”是没有意义,必须要指明是沿哪个方向的“方向导数”,每一个$\vb*{l}$的取值就将对应一个不同的方向导数$\pdv*{u}{\vb*{l}}$场。

方向导数$\pdv*{u}{\vb*{l}}$,简单的说,就表征了$u(\vb*{r})$沿$\vb*{l}$方向的变换快慢。

方向导数的定义是无关坐标系的,但其具体形式与坐标系有关,在直角坐标系中
\begin{Equation}
    \pdv{u}{l}=\pdv{u}{x}\cos\alpha+\pdv{u}{y}\cos\beta+\pdv{u}{z}\cos\gamma
\end{Equation}
其中$\cos\alpha,\cos\beta,\cos\gamma$分别是$\vb*{l}$的方向余弦。

标量场的梯度的概念又是怎么来的呢?试想,在场函数$u(\vb*{r})$的每个点处,沿每个$\vb*{l}$方向都会有不同的方向导数,我们可以对于每个点$\vb*{r}$选取该点处最大的那一方向导数$\max{\pdv*{u}{\vb*{l}}}$,并记使得$\pdv*{u}{\vb*{l}}$最大的$\vb*{l}$的单位矢量为$\vb*{e}_\text{n}$,而所谓梯度,就是指两者相乘所构成的矢量场。
\begin{BoxDefinition}[标量场的梯度]
    定义标量场$u(\vb*{r})$的\uwave{梯度}(Gradient)
    \begin{Equation}
        \grad u=\Grad u=\vb*{e}_\text{n}\max\pdv{u}{l}
    \end{Equation}
    其中,$\vb*{e}_\text{n}$代表对于每个点$r$使得$\pdv*{u}{l}$最大的$\vb*{l}$的单位矢量。
\end{BoxDefinition}

梯度是矢量场,其大小表示该点的最大的变化率,其方向表示该点最大的变化率的方向。

梯度的定义是无关坐标系的,但梯度的具体形式则与坐标系有关,由于这一部分内容我们在微积分和数学物理方法中已经非常熟悉了,以下我们直接列出梯度在各个坐标系中的形式。

\begin{BoxFormula}[直角坐标系的梯度]
    在直角坐标系下,梯度的形式是
    \begin{Equation}
        \grad u=\vb*{e}_x\pdv{u}{x}+\vb*{e}_y\pdv{u}{y}+\vb*{e}_z\pdv{u}{z}
    \end{Equation}
\end{BoxFormula}
\begin{BoxFormula}[柱坐标系的梯度]
    在柱坐标系下,梯度的形式是
    \begin{Equation}
        \grad u=\vb*{e}_\rho\pdv{u}{\rho}+\vb*{e}_\phi\frac{1}{\rho}\pdv{u}{\phi}+\vb*{e}_z\pdv{u}{z}
    \end{Equation}
\end{BoxFormula}
\begin{BoxFormula}[球坐标系的梯度]
    在球坐标系下,梯度的形式是
    \begin{Equation}
        \grad u=\vb*{e}_r\pdv{u}{r}+\vb*{e}_\theta\frac{1}{r}\pdv{u}{\theta}+\vb*{e}_\phi\frac{1}{r\sin\theta}\pdv{u}{\phi}
    \end{Equation}
\end{BoxFormula}
这里$\grad$称为\uwave{哈密顿算符},常读作Del或Nabla,它在直角坐标系下的形式为
\begin{Equation}
    \grad=\vb*{e}_x\pdv{x}+\vb*{e}_y\pdv{y}+\vb*{e}_z\pdv{z}
\end{Equation}
由于$\grad$具有矢量和微分的双重性质,故又称为\uwave{矢性微分算符},引入$\grad$的意义在于其可以在直角坐标系下,将梯度、散度、旋度统一表示为$\grad u~,\div\vb*{A},~\curl\vb*{A}$,即将算符$\grad$视为一个矢量分别与场函数进行数乘、点乘、叉乘。不过,这种想法只能适用于直角坐标系,因为\xref{sec:正交曲面坐标系}告诉我们,在柱坐标系和球坐标系下,两个矢量的点乘和叉乘并没有完善的定义,因此,这种情况下再谈论$\grad$与矢量场函数$\vb*{A}$的点乘和叉乘就没有意义了。总的说,$\grad u~,\div\vb*{A},~\curl\vb*{A}$的记法是源于直角坐标,在直角坐标下也确实可以将梯度、散度、旋度理解为$\grad$与场函数的数乘、点乘、叉乘。但是,柱坐标和球坐标中,这种理解并不适用.柱坐标和球坐标中只是继承了用$\grad u~,\div\vb*{A},~\curl\vb*{A}$表示梯度、散度、旋度的记法,并没有继承这种记法的内涵,换言之,柱坐标和球坐标中的$\grad u~,\div\vb*{A},~\curl\vb*{A}$只不过是$\Grad u,~\Div u,~\Curl u$的别称罢了。

\subsection{场点与源点}
在这一小节,我们将会研究两个计算梯度的示例,由此引出场点和源点的概念,这里设
\begin{Equation}[场点和源点]
    \vb*{R}=\vb*{e}_x(x-x')+\vb*{e}_y(y-y')+\vb*{e}_z(z-z')\qquad R=\abs{\vb*{R}}
\end{Equation}
关于\xref{eq:场点和源点},做两点说明
\begin{itemize}
    \item 此处$(x,y,z)$称为\uwave{场点},表示场分布的位置,是变量。
    \item 此处$(x',y',z')$称为\uwave{源点},表示场源分布的位置,是常量。
\end{itemize}
\begin{BoxFormula}[距离正比的梯度]
    若标量场函数正比于至源点$(x',y',z')$半径$R$,则其梯度为
    \begin{Equation}
        \grad R=\frac{\vb*{R}}{R}
    \end{Equation}
\end{BoxFormula}
\begin{Proof}
    根据\fancyref{fml:直角坐标系的梯度}
    \begin{Equation}
        \grad R=\vb*{e}_x\pdv{R}{x}+\vb*{e}_y\pdv{R}{y}+\vb*{e}_z\pdv{R}{z}
    \end{Equation}
    注意到
    \begin{Equation}
        R=\abs{\vb*{R}}=[(x-x')^2+(y-y')^2+(z-z')^2]^{\frac{1}{2}}
    \end{Equation}
    因此
    \begin{Equation}
        \grad R=\frac{2\vb*{e}_x(x-x')+2\vb*{e}_y(y-y')+2\vb*{e}_z(z-z')}{2[(x-x')^2+(y-y')^2+(z-z')^2]^{\frac{1}{2}}}
    \end{Equation}
    将分子分母上的$2$约掉,将$\vb*{R},R$代回
    \begin{Equation}*
        \grad R=\frac{\vb*{R}}{R}\qedhere
    \end{Equation}
\end{Proof}

\begin{BoxFormula}[距离反比的梯度]
    若标量场函数反比于至源点$(x',y',z')$半径$R$,则其梯度为
    \begin{Equation}
        \grad R^{-1}=-\frac{\vb*{R}}{R^3}
    \end{Equation}
\end{BoxFormula}

\begin{Proof}
    根据\fancyref{fml:直角坐标系的梯度}
    \begin{Equation}
        \grad R^{-1}=\vb*{e}_x\pdv{R^{-1}}{x}+\vb*{e}_y\pdv{R^{-1}}{y}+\vb*{e}_z\pdv{R^{-1}}{z}
    \end{Equation}
    注意到
    \begin{Equation}
        R^{-1}=\abs{\vb*{R}}^{-1}=[(x-x')^2+(y-y')^2+(z-z')^2]^{-\frac{1}{2}}
    \end{Equation}
    因此
    \begin{Equation}
        \grad R^{-1}=-\frac{2\vb*{e}_x(x-x')+2\vb*{e}_y(y-y')+2\vb*{e}_z(z-z')}{2[(x-x')^2+(y-y')^2+(z-z')^2]^{\frac{3}{2}}}
    \end{Equation}
    将分子分母上的$2$约掉,将$\vb*{R},R$代回
    \begin{Equation}*
        \grad R^{-1}=-\frac{\vb*{R}}{R^3}\qedhere
    \end{Equation}
\end{Proof}

我们说,场点$(x,y,z)$是变量,源点$(x',y',z')$是作为参数的常量,因此以上我们的梯度运算都是对场点进行的,但其实很多时候,取决于我们怎么看,常量和变量是可以相互转换的。

我们引入以下记号
\begin{Equation}
    \qquad\qquad\qquad
    \grad=\vb*{e}_x\pdv{x}+\vb*{e}_y\pdv{y}+\vb*{e}_z\pdv{z}\qquad
    \grad'=\vb*{e}_x\pdv{x'}+\vb*{e}_y\pdv{y'}+\vb*{e}_z\pdv{z'}
    \qquad\qquad\qquad
\end{Equation}
这里,$\grad$是对场点的微分(源点视为常量),$\grad'$是对源点的微分(场点视为场量)。\goodbreak

这两者的存在于电磁学而言都是非常必要的
\begin{itemize}
    \item 研究一个电荷对不同位置的影响(场点变化),使用场点坐标$(x,y,z)$和$\grad$算符。
    \item 研究多个电荷对同一位置的影响(源点变化),使用源点坐标$(x',y',z')$和$\grad'$算符。
\end{itemize}
而事实是,\empx{场点坐标和源点坐标是可以相互转化的},就梯度而言,两者仅相差一个负号。

\begin{BoxFormula}[场点坐标和源点坐标的转化]
    场点坐标和源点坐标下,梯度运算的结果相差一个负号
    \begin{Equation}
        \grad f(R)=-\grad' f(R)
    \end{Equation}
\end{BoxFormula}
\begin{Proof}
    在场点坐标下运算
    \begin{Equation}
        \grad f(R)=\dv{f(R)}{R}\grad f(R)=\dv{f(R)}{R}\frac{\vb*{R}}{R}
    \end{Equation}
    在源点坐标下运算
    \begin{Equation}
        \grad' f(R)=\dv{f(R)}{R}\grad' f(R)=-\dv{f(R)}{R}\frac{\vb*{R}}{R}
    \end{Equation}
    为何如此呢?关键是因为
    \begin{Equation}
        R=[(x-x')^2+(y-y')^2+(z-z')^2]^{\frac{1}{2}}
    \end{Equation}
    即$(x,y,z),(x',y',z')$前的负号相反,若交换两者的地位,求导结果自然也相差一个负号。
\end{Proof}
这种场点坐标和源点坐标的相互转化,在电磁场中是十分有用的。