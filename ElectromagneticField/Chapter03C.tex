\section{静电场的能量}
静电场最基本的特征是对电荷有作用力,这表明静电场具有能量。我们或许会认为静电场的能量来自场源电荷,但是大量实践表明,更合理的解释是,静电场的能量是定域在空间中的。

\begin{BoxFormula}[静电场的能量]
    静电场的能量定域在电场不为零的空间中,能量密度满足
    \begin{Equation}
        w_\text{e}=\frac{1}{2}D\cdot E
    \end{Equation}
\end{BoxFormula}

\begin{Proof}
    设电荷分布区域为$V$,其边界曲面记为$S$,现在考虑电荷系统从零开始被充电,当充电完毕后,电荷分布为$\rho$,电势分布为$\varphi$。根据\xref{sec:静电场与电容}的内容,我们知道充电过程中电荷分布和电势分布将以同一比例因子$\alpha$增加,换言之,若充电过程中,某一时刻的电荷分布为$\alpha\rho$,则该时刻的电势分布即为$\alpha\varphi$。电荷与电势的乘积就是能量,故$\alpha$至$\alpha+\dd{\alpha}$过程中,外电源做功为
    \begin{Equation}&[1]
        \dd{W_\text{e}}=\Itnt[V](\alpha\varphi)\rho\dd{\alpha}\dd{V}
    \end{Equation}
    这里,$\alpha\varphi$即当前的电势分布,$\rho\dd{\alpha}$即该微元过程增加的电荷密度,在$V$上积分。

    而讨论整个充电过程外电源的做功,即对$\alpha$在$[0,1]$上积分
    \begin{Equation}&[2]
        W_\text{e}=\Int[0][1]\alpha\dd{\alpha}\Itnt[V]\rho\varphi\dd{V}
    \end{Equation}
    这很容易求出
    \begin{Equation}&[3]
        W_\text{e}=\frac{1}{2}\Itnt[V]\rho\varphi\dd{V}
    \end{Equation}
    这里算得的$W_\text{e}$就是静电场的能量,下面我们将其以电场矢量表示。

    根据\fancyref{eqt:麦克斯韦方程组},我们可以在\xrefpeq{3}中代入$\div\vb*{D}=\rho$
    \begin{Equation}&[4]
        W_\text{e}=\frac{1}{2}\Itnt[V]\qty(\div\vb*{D})\varphi\dd{V}
    \end{Equation}
    根据矢量分析的公式
    \begin{Equation}&[5]
        \div(\varphi\vb*{D})=\grad\varphi\cdot\vb*{D}+(\div\vb*{D})\varphi
    \end{Equation}
    这样\xrefpeq{4}就可以被改写为
    \begin{Equation}&[6]
        W_\text{e}=\frac{1}{2}\Itnt[V]\div(\varphi\vb*{D})\dd{V}+\frac{1}{2}\Itnt[V]\grad\varphi\cdot\vb*{D}\dd{V}
    \end{Equation}
    就\xrefpeq{6}的第一项运用\fancyref{thm:散度定理},就第二项运用$\vb*{E}=-\grad\vb*{E}$
    \begin{Equation}&[7]
        W_\text{e}=\frac{1}{2}\Isot[S]\varphi\vb*{D}\cdot\dd{\vb*{S}}+\frac{1}{2}\Itnt[V]\vb*{E}\cdot\vb*{D}\dd{V}
    \end{Equation}
    我们可以令$V\to\R^3$,即使体积分转为对整个空间的积分,因为反正只有那些有电荷分布的空间才对积分有贡献。而当$V\to\R^3$时闭曲面$S$的表面积也将无限扩大,此时,只要电荷是分布在有限区域内,那么当$S$无限扩大时,有限区域内的电荷就都可以近似为一个点电荷。

    而对于点电荷,我们有
    \begin{Equation}&[8]
        \varphi\propto\frac{1}{r}\qquad
        |\vb*{D}|\propto\frac{1}{r^2}
    \end{Equation}
    即
    \begin{Equation}&[9]
        |\varphi\vb*{D}|\propto\frac{1}{r^3}
    \end{Equation}
    因此,当$S$无限扩大,即$r\to\infty$时
    \begin{Equation}&[10]
        \frac{1}{2}\Isot[S]\varphi\vb*{D}\cdot\dd{\vb*{S}}\propto\frac{1}{r^3}r^2\propto\frac{1}{r}\to 0
    \end{Equation}
    这样一来,\xrefpeq{7}就可以简化为
    \begin{Equation}
        W_\text{e}=\frac{1}{2}\Itnt[\R^3]\vb*{E}\cdot\vb*{D}\dd{V}
    \end{Equation}
    因此,能量密度就是
    \begin{Equation}
        w_\text{e}=\frac{1}{2}\vb*{E}\cdot\vb*{D}
    \end{Equation}
\end{Proof}