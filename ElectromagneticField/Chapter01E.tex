\section{无旋场的标量位}

\subsection{无旋场的定义}
\begin{BoxDefinition}[无旋场的定义]
    若一个矢量场$\vb*{F}$的旋度处处为零,即
    \begin{Equation}
        \curl\vb*{F}=\vb*{0}
    \end{Equation}
    则称该矢量场为无旋场,它完全由散度源产生。
\end{BoxDefinition}
根据\fancyref{thm:旋度定理},无旋场$\vb*{F}$沿任意闭合路径$C$的环流为$0$,即
\begin{Equation}
    \Ilot[C]\vb*{F}\cdot\dd{\vb*{l}}=\Isnt[S]\curl\vb*{F}\cdot\dd{\vb*{S}}=0
\end{Equation}

\subsection{无旋场的产生}
现在的问题是,什么样的场一定是无旋场$\vb*{F}$呢?下面的结论将给我们答案。
\begin{BoxProperty}[标量场的梯度无旋]
    标量场的梯度,必为无旋场(梯度场无旋)
    \begin{Equation}
        \curl(\grad u)=\vb*{0}
    \end{Equation}
\end{BoxProperty}
\begin{Proof}
    % 这个结论在数学物理方法中我们已经证明过了,但这里换一种更优雅的方式来完成。

    任取一个空间曲面$S$,将$\curl(\grad u)$在$S$上积分
    \begin{Equation}&[1]
        I=\Isnt[S]\curl(\grad u)\cdot\dd{\vb*{S}}
    \end{Equation}
    设空间曲面$S$的边界曲线为$C$,根据\fancyref{thm:旋度定理}
    \begin{Equation}&[2]
        I=
        \Ilot[C]\grad u\cdot\dd{\vb*{l}}=
        \Ilot[C]\grad u\cdot\vb*{e}_l\dd{l}
    \end{Equation}
    运用梯度和方向导数间的关系$\grad u\cdot\vb*{e}_l=\pdv*{u}{l}$
    \begin{Equation}&[3]
        I=
        \Ilot[C]\pdv{u}{l}\dd{l}=\Ilot[C]\dd{u}=0
    \end{Equation}
    将该积分结果代回\xrefpeq{1},考虑到曲面$S$的任意性
    \begin{Equation}*
        \curl(\grad u)=0\qedhere
    \end{Equation}
\end{Proof}

\subsection{无旋场的标势}
既然标量场的梯度为无旋场,因此,若$\vb*{F}$是无旋场,那么$\vb*{F}$必然可以表示为标量场的梯度。
\begin{BoxDefinition}[无旋场的标势]
    若$\vb*{F}$是无旋场,则必然存在标量函数$u$使得
    \begin{Equation}
        \vb*{F}=-\grad u
    \end{Equation}
    该标量函数$u$就是无旋场$\vb*{F}$的\uwave{标量位}或\uwave{标势}(Scalar Potential)。
\end{BoxDefinition}

简而言之,\empx{无旋场是其标势的负梯度},这里的负号来源于物理上$\vb*{E}=-\grad\phi$的习惯。

在电磁场中,我们往往会将对某个无旋场性质的讨论,转化为对无旋场的标势性质的讨论。\goodbreak

无旋场$\vb*{F}$的旋度显然为零,这就是无旋场的定义
\begin{Equation}
    \curl\vb*{F}=\vb*{0}
\end{Equation}
无旋场$\vb*{F}$的散度则是有值的,设为$g$
\begin{Equation}
    \div\vb*{F}=g
\end{Equation}
在上式中代入\fancyref{def:无旋场的标势}
\begin{Equation}
    \div(\grad u)=-g
\end{Equation}
这种“先求梯度,再求散度”的形式我们很熟悉(参见\xrefpeq{fml:标量拉普拉斯的转化}),它可以改写为
\begin{Equation}
    (\div\grad)u=-g
\end{Equation}
这里$\div\grad$称为\uwave{标量拉普拉斯算符},记为$\laplacian$,由此,我们就可以得到一个重要的事实。
\begin{BoxFormula}[标量泊松方程]
    无旋场$\vb*{F}$的散度性质,可以等价的用无旋场$\vb*{F}$的标势的标量泊松方程表示
    \begin{Equation}
        \laplacian u=-g
    \end{Equation}
    特别的,在$g=0$的部分,这就将给出标量拉普拉斯方程
    \begin{Equation}
        \laplacian u=0
    \end{Equation}
    其中,$g$为$\vb*{F}$的散度,$u$为$\vb*{F}$的标势,即
    \begin{Equation}
        \div\vb*{F}=g\qquad \curl\vb*{F}=\vb*{0}\qquad \vb*{F}=-\grad u
    \end{Equation}
\end{BoxFormula}

前面将梯度的散度转化为标量拉普拉斯的过程,其实也是一个矢量分析公式。
\begin{BoxFormula}[标量拉普拉斯的转化]
    标量拉普拉斯可以作以下转化
    \begin{Equation}
        \laplacian u=\div(\grad u)
    \end{Equation}
    即,标量拉普拉斯,就是梯度的散度。
\end{BoxFormula}\goodbreak

这里列出了拉普拉斯算符在各个坐标系的形式
\begin{BoxFormula}[直角坐标系的拉普拉斯]
    在直角坐标系下,拉普拉斯的形式是
    \begin{Equation}
        \laplacian u=\pdv[2]{u}{x}+\pdv[2]{u}{y}+\pdv[2]{u}{z}
    \end{Equation}
\end{BoxFormula}
\begin{BoxFormula}[柱坐标系的拉普拉斯]
    在柱坐标系下,拉普拉斯的形式是
    \begin{Equation}
        \laplacian u=\frac{1}{\rho}\pdv{\rho}\qty(\rho\pdv{u}{\rho})+\frac{1}{\rho^2}\pdv[2]{u}{\phi}+\pdv[2]{u}{z}
    \end{Equation}
\end{BoxFormula}
\begin{BoxFormula}[球坐标系的拉普拉斯]*
    在球坐标系下,拉普拉斯的形式是
    \begin{Equation}
        \qquad\qquad\qquad
        \laplacian u=\frac{1}{r^2}\pdv{r}\qty(r^2\pdv{u}{r})+\frac{1}{r^2\sin\theta}\pdv{\theta}\qty(\sin\theta\pdv{u}{\theta})+\frac{1}{r^2\sin^2\theta}\pdv[2]{u}{\phi}
        \qquad\qquad\qquad
    \end{Equation}
\end{BoxFormula}