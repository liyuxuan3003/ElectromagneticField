\section{电磁场的边界条件}
在解决实际的电磁场工程问题中,通常要设计到不同电磁参数的物质所构成的相邻区域。由于电磁参数在物质分界面上发生了变化,导致电磁场矢量也会随之发生突变,使\xref{sec:麦克斯韦方程组}中的麦克斯韦方程的微分形式在分界面上失去意义。这种情况下,为了求解各个区域中的电磁场问题,必须要知道在两种物质分界面两侧的电磁场量间的关系,我们称为\uwave{电磁场的边界条件}。

边界条件在求解电磁问题的过程中占据非常重要的地位,这是因为只有使得麦克斯韦方程组的通解适合于某个包含给定的区域及其边界条件,这个解,才是有实际意义的唯一解。

边界条件如何求呢?关键在于利用麦克斯韦方程组的积分形式,因为积分形式的麦克斯韦方程组不受场量连续与否的影响,在边界上也是成立的。本节就将以这种方式导出边界条件。

在接下来两个小节中,我们均以$\vb*{e}_\text{n}$表示分界面上,由介质1指向介质2的法向矢量。

\subsection{切向边界条件}
\begin{BoxFormula}[磁场强度的边界条件]
    磁场强度的边界条件满足
    \begin{Equation}
        \vb*{e}_\text{n}\times(\vb*{H}_1-\vb*{H}_2)=\vb*{J}_{S}
    \end{Equation}
    这表明,磁场强度的切向分量在分界面发生跃变,跃变量为面电流密度。
\end{BoxFormula}
\begin{Proof}
    如\xref{fig:介质分界面上的小矩形回路}所示,设想,我们现在考察介质分界面上的某一点$P$处的情况,我们在点$P$的周围取一个小矩形回路$ABCD$,设其所围面积是$\delt{S}$。矩形回路$AB$和$CD$两条边分别位于分界面的两侧并与之平行,其长度$AB=CD=\delt{l}$很小,其间距$BC=DA=\delt{h}\to 0$趋于零。
    \begin{Figure}[介质分界面上的小矩形回路]
        \includegraphics[scale=0.9]{build/Chapter02G_01.fig.pdf}
    \end{Figure}
    这里我们主要涉及到三个单位矢量
    \begin{itemize}
        \item 单位矢量$\vb*{e}_\text{n}$表示两种介质的边界曲面上的法向矢量。
        \item 单位矢量$\vb*{e}_\text{p}$表示矩形区域面积$\delt{S}$的法向矢量(图中未绘出)。
        \item 单位矢量$\vb*{e}_\text{t}$表示沿矩形$AB$方向上的单位矢量。
    \end{itemize}
    第一步,我们运用\fancyref{eqt:麦克斯韦方程组的积分形式}
    \begin{Equation}&[1]
        \Ilot[C]\vb*{H}\cdot\dd{\vb*{l}}=\Isnt[\delt{S}]\vb*{J}\cdot\dd{\vb*{S}}+\Isnt[\delt{S}]\pdv{\vb*{D}}{t}\cdot\dd{\vb*{S}}
    \end{Equation}
    这里$C$就是小矩形回路$ABCD$,我们前面提到过,我们要令$\delt{h}\to 0$。

    由于电位移$\vb*{D}$是连续变化的,因此$\pdv*{\vb*{D}}{t}$应当是有限值,故
    \begin{Equation}&[2]
        \Lim[\delt{h}\to 0]\Isnt[\delt{S}]\pdv{\vb*{D}}{t}\cdot\dd{\vb*{S}}=0
    \end{Equation}
    由于当$\delt{h}\to 0$时,真正对$\delt{S}$的电流密度通量有贡献的必然是分界面上的面电流$\vb*{J}_{S}$,故
    \begin{Equation}&[3]
        \Lim[\delt{h}\to 0]\Isnt[\delt{S}]\vb*{J}\cdot\dd{\vb*{S}}=\Ilnt[\delt{l}]\vb*{J}_S\cdot\vb*{e}_\text{p}\dd{l}
    \end{Equation}
    这样一来,将\xrefpeq{2}和\xrefpeq{3}代回\xrefpeq{1}
    \begin{Equation}&[4]
        \Lim[\delt{h}\to 0]\Ilot[C]\vb*{H}\cdot\dd{\vb*{l}}=\Ilnt[\delt{l}]\vb*{J}_S\cdot\vb*{e}_\text{p}\dd{l}
    \end{Equation}
    第二步,我们将$\vb*{H}$沿$C$的积分拆为四段
    \begin{Equation}&[5]
        \Ilot[C]\vb*{H}\cdot\dd{\vb*{l}}=\Ilnt[AB]\vb*{H}\cdot\dd{\vb*{l}}
        +\Ilnt[BC]\vb*{H}\cdot\dd{\vb*{l}}
        +\Ilnt[CD]\vb*{H}\cdot\dd{\vb*{l}}
        +\Ilnt[DA]\vb*{H}\cdot\dd{\vb*{l}}
    \end{Equation}
    由于磁场强度$\vb*{H}$应当是有限值,故
    \begin{Equation}&[6]
        \Lim[\delt{h}\to 0]\qty[
            \Ilnt[BC]\vb*{H}\cdot\dd{\vb*{l}}+
            \Ilnt[DA]\vb*{H}\cdot\dd{\vb*{l}}
        ]=0
    \end{Equation}
    而同时,随着$\delt{h}\to 0$,路径$AB,CD$都缩到同一条路径$\delt{l}$上,但在边界两侧沿相反方向
    \begin{Equation}&[7]
        \Lim[\delt{h}\to 0]\qty[
            \Ilnt[AB]\vb*{H}\cdot\dd{\vb*{l}}+
            \Ilnt[CD]\vb*{H}\cdot\dd{\vb*{l}}
        ]=\Ilnt[\delt{l}](\vb*{H}_1-\vb*{H}_2)\cdot\vb*{e}_\text{t}\dd{l}
    \end{Equation}
    这样一来,将\xrefpeq{6}和\xrefpeq{7}代入\xrefpeq{5}
    \begin{Equation}&[8]
        \Lim[\delt{h}\to 0]\Ilot[c]\vb*{H}\cdot\dd{\vb*{l}}=\Ilnt[\delt{l}](\vb*{H}_1-\vb*{H}_2)\cdot\vb*{e}_\text{t}\dd{l}
    \end{Equation}
    由此,联立\xrefpeq{4}和\xrefpeq{8},得到
    \begin{Equation}&[9]
        \Ilnt[\delt{l}]\vb*{J}_{S}\cdot\vb*{e}_\text{p}\dd{l}=\Ilnt[\delt{l}](\vb*{H}_1-\vb*{H}_2)\cdot\vb*{e}_\text{t}\dd{l}
    \end{Equation}
    由于$\delt{l}$很小时,可以认为$\vb*{H}_1,\vb*{H}_2,\vb*{J}_S$都是均匀的,故
    \begin{Equation}&[10]
        \vb*{J}_{S}\cdot\vb*{e}_\text{p}\delt{l}=(\vb*{H}_1-\vb*{H}_2)\cdot\vb*{e}_\text{t}\delt{l}
    \end{Equation}
    两端约去$\delt{l}$
    \begin{Equation}&[11]
        \vb*{J}_{S}\cdot\vb*{e}_\text{p}=(\vb*{H}_1-\vb*{H}_2)\cdot\vb*{e}_\text{t}
    \end{Equation}
    由于$\vb*{e}_\text{t}=\vb*{e}_\text{p}\times\vb*{e}_\text{n}$,代入\xrefpeq{11}右端,并使用\fancyref{fml:标量三重积的轮换对称性}
    \begin{Equation}&[12]
        \qquad\qquad\quad
        (\vb*{H}_1-\vb*{H}_2)\cdot\vb*{e}_\text{t}=(\vb*{H}_1-\vb*{H}_2)\cdot(\vb*{e}_\text{p}\times\vb*{e}_\text{n})=\qty[\vb*{e}_\text{n}\times(\vb*{H}_1-\vb*{H}_2)]\cdot\vb*{e}_\text{p}
        \qquad\qquad\quad
    \end{Equation}
    将\xrefpeq{12}代入\xrefpeq{11}
    \begin{Equation}&[13]
        \vb*{J}_{S}\cdot\vb*{e}_\text{p}=\qty[\vb*{e}_\text{n}\times(\vb*{H}_1-\vb*{H}_2)]\cdot\vb*{e}_\text{p}
    \end{Equation}
    约去$\vb*{e}_\text{p}$即得
    \begin{Equation}*
        \vb*{e}_\text{n}\times(\vb*{H}_1-\vb*{H}_2)=\vb*{J}_{S}\qedhere
    \end{Equation}
\end{Proof}
% 由此可见,分界面上存在自由面电流时,分界面两侧的磁场强度矢量$\vb*{H}$的切向分量是不连续的。这是因为电流是磁场强度$\vb*{H}$的涡旋源,在分界面上的面电流在分界面两侧产生的磁场强度矢量,在切向是相反的,这就导致了$\vb*{H}$在切向上的跃变。

\begin{BoxFormula}[电场强度的边界条件]
    电场强度的边界条件满足
    \begin{Equation}
        \vb*{e}_\text{n}\times(\vb*{E}_1-\vb*{E}_2)=\vb*{0}
    \end{Equation}
    这表明,电场强度的切向分量在分界面是连续的。
\end{BoxFormula}

\begin{Proof}
    如\xref{fig:介质分界面上的小矩形回路},只不过图中的磁场强度$\vb*{H}_1,\vb*{H}_2$应当改换为电场强度$\vb*{E}_1,\vb*{E}_2$。

    第一步,我们运用\fancyref{eqt:麦克斯韦方程组的积分形式}
    \begin{Equation}&[1]
        \Ilot[C]\vb*{E}\cdot\dd{\vb*{l}}=\Isnt[S]\pdv{\vb*{B}}{t}\cdot\dd{\vb*{S}}
    \end{Equation}

    由于磁感应强度$\vb*{B}$是连续变化的,因此$\pdv*{B}{t}$应当是有限值,故
    \begin{Equation}&[2]
        \Lim[\delt{h}\to 0]\Isnt[S]\pdv{\vb*{B}}{t}\cdot\dd{\vb*{S}}=0
    \end{Equation}
    因而
    \begin{Equation}&[3]
        \Lim[\delt{h}\to 0]\Ilot[C]\vb*{E}\cdot\dd{\vb*{l}}=0
    \end{Equation}
    第二步,根据前面的经验
    \begin{Equation}&[4]
        \Lim[\delt{h}\to 0]\Ilot[C]\vb*{E}\cdot\dd{\vb*{l}}=\Ilnt[\delt{l}](\vb*{E}_1-\vb*{E}_2)\cdot\vb*{e}_\text{t}\dd{l}
    \end{Equation}
    联立\xrefpeq{3}和\xrefpeq{4},再效仿前面的转化方法,最终得到
    \begin{Equation}*
        \vb*{e}_\text{n}\times(\vb*{E}_1-\vb*{E}_2)=\vb*{0}\qedhere
    \end{Equation}
\end{Proof}

\subsection{法向边界条件}
\begin{BoxFormula}[电位移矢量的边界条件]
    电位移矢量的边界条件满足
    \begin{Equation}
        \vb*{e}_\text{n}\cdot(\vb*{D}_1-\vb*{D}_2)=\rho_\text{S}
    \end{Equation}
    这表明,电位移矢量在法向分量在分界面发生跃变,跃变量为面电荷密度。
\end{BoxFormula}

\begin{Proof}
    如\xref{fig:介质分界面上的小圆柱形闭合曲面}所示,设想,我们现在考察介质分界面上的某一点$P$处的情况,我们在点$P$的周围取一个底面积$\delt{S}$很小,高$\delt{h}\to 0$的小圆柱形闭合曲面,上下底面平行位于分界面两侧。
    \begin{Figure}[介质分界面上的小圆柱形闭合曲面]
        \includegraphics[scale=0.9]{build/Chapter02G_02.fig.pdf}
    \end{Figure}

    第一步,我们运用\fancyref{eqt:麦克斯韦方程组的积分形式}
    \begin{Equation}&[1]
        \Isot[S]\vb*{D}\cdot\dd{\vb*{S}}=\Itnt[V]\rho\dd{V}
    \end{Equation}
    这里$S$就是小圆柱形闭合曲面,而$V$是该闭合曲面包围的空间区域,我们要令$\delt{h}\to 0$。

    由于当$\delt{h}\to 0$时,真正对$V$中的电荷量有贡献的必然是分界面上的面电荷$\rho_S$,故
    \begin{Equation}&[2]
        \Lim[\delt{h}\to 0]\Itnt[V]\rho\dd{V}=\Isnt[\delt{S}]\rho_S\dd{S}
    \end{Equation}
    这样一来,将\xrefpeq{2}代入\xrefpeq{1}
    \begin{Equation}&[3]
        \Lim[\delt{h}\to 0]\Isot[S]\vb*{D}\cdot\dd{\vb*{S}}=\Isnt[\delt{S}]\rho_S\dd{S}
    \end{Equation}
    第二步,我们将$\vb*{D}$在闭合圆柱面上的积分拆分为底面和侧面
    \begin{Equation}&[4]
        \Isot[S]\vb*{D}\cdot\dd\vb*{S}=
        \Isnt[\text{上底面}]\vb*{D}\cdot\dd{\vb*{S}}+
        \Isnt[\text{下底面}]\vb*{D}\cdot\dd{\vb*{S}}+
        \Isnt[\text{侧面}]\vb*{D}\cdot\dd{\vb*{S}}
    \end{Equation}
    由于电位移矢量$\vb*{D}$应当是有限值,故
    \begin{Equation}&[5]
        \Lim[\delt{h}\to 0]\Isnt[\text{侧面}]\vb*{D}\cdot\dd{\vb*{S}}=0
    \end{Equation}
    而同时,随着$\delt{h}\to 0$,上下底面都缩到$\delt{S}$上,但两者朝向相反
    \begin{Equation}&[6]
        \Lim[\delt{h}\to 0]\qty[\Isnt[\text{上底面}]\vb*{D}\cdot\dd{\vb*{S}}+
        \Isnt[\text{下底面}]\vb*{D}\cdot\dd{\vb*{S}}]=\Isnt[\delt{S}](\vb*{D}_1-\vb*{D}_2)\cdot\vb*{e}_\text{n}\dd{S}\hspace*{-0.1cm}
    \end{Equation}
    这样一来,将\xrefpeq{5}和\xrefpeq{6}代入\xrefpeq{4}
    \begin{Equation}&[7]
        \Lim[\delt{h}\to 0]\Isot[S]\vb*{D}\cdot\dd{\vb*{S}}=\Isnt[\delt{S}](\vb*{D}_1-\vb*{D}_2)\cdot\vb*{e}_\text{n}\dd{S}
    \end{Equation}
    由此,联立\xrefpeq{3}和\xrefpeq{7},得到
    \begin{Equation}
        \Isnt[\delt{S}]\rho_S\dd{S}=\Isnt[\delt{S}](\vb*{D}_1-\vb*{D}_2)\cdot\vb*{e}_\text{n}\dd{S}
    \end{Equation}
    即
    \begin{Equation}*
        \vb*{e}_\text{n}\cdot(\vb*{D}_1-\vb*{D}_2)=\rho_\text{S}\qedhere
    \end{Equation}
\end{Proof}

\begin{BoxFormula}[磁感应强度的边界条件]
    磁感应强度的边界条件满足
    \begin{Equation}
        e_\text{n}\cdot(\vb*{B}_1-\vb*{B}_2)=0
    \end{Equation}
    这表明,磁感应强度的法向分量在分界面是连续的。
\end{BoxFormula}

\begin{Proof}
    如\xref{fig:介质分界面上的小圆柱形闭合曲面},只不过图中的电位移矢量$\vb*{D}_1,\vb*{D}_2$应当改换为磁感应强度$\vb*{B}_1,\vb*{B}_2$。

    第一步,我们应用\fancyref{eqt:麦克斯韦方程组的积分形式}
    \begin{Equation}&[1]
        \Isot[S]\vb*{B}\cdot\dd{\vb*{S}}=0
    \end{Equation}
    因而,很显然的
    \begin{Equation}&[2]
        \Lim[\delt{h}\to 0]\Isot[S]\vb*{B}\cdot\dd{\vb*{S}}=0
    \end{Equation}
    第二步,根据前面的经验
    \begin{Equation}&[3]
        \Lim[\delt{h}\to 0]\Isot[S]\vb*{B}\cdot\dd{\vb*{S}}=\Isnt[\delt{S}](\vb*{B}_1-\vb*{B}_2)\cdot\vb*{e}_\text{n}\dd{S}
    \end{Equation}
    联立\xrefpeq{2}和\xrefpeq{3},得到
    \begin{Equation}*
        e_\text{n}\cdot(\vb*{B}_1-\vb*{B}_2)=0\qedhere
    \end{Equation}
\end{Proof}

在此,让我们来总结一下四个电磁场矢量的边界条件
\begin{itemize}
    \item \xref{fml:磁场强度的边界条件}指出,矢量$\vb*{H}$的切向分量在边界不连续,发生面电流密度的跃变。
    \item \xref{fml:电场强度的边界条件}指出,矢量$\vb*{E}$的切向分量在边界连续。
    \item \xref{fml:磁感应强度的边界条件}指出,矢量$\vb*{B}$的法向分量在边界连续。
    \item \xref{fml:电位移矢量的边界条件}指出,矢量$\vb*{D}$的法向分量在边界不连续,发生面电荷密度的跃变。
\end{itemize}
在这里,一个有趣的问题是,为什么面电流和面电荷导致的跃变分别是切向和法向?
\begin{itemize}
    \item 面电流密度是$\vb*{H}$的涡旋源,涡旋源在分界面两侧产生的场,切向相反。
    \item 面电荷密度是$\vb*{D}$的通量源,通量源在分界面两侧产生的场,法向相反。
\end{itemize}
还有一个值得注意的问题是,由于
\begin{Equation}
    \vb*{J}_{S}=\Lim[\delt{h}\to 0]\vb*{J}\delt{h}=\Lim[\delt{h}\to 0]\sigma\vb*{E}\delt{h}
\end{Equation}
由于电场强度$\vb*{E}$为有限值,而$\delt{h}\to 0$,因此,面电流密度$\vb*{J}_{S}$若要有非零值,就必须要使得电导率$\sigma$为无穷大,换言之,电导率有限的介质中,矢量$\vb*{H}$的切向分量在边界将是连续的。

\subsection{理想导体的边界条件}
\uwave{理想导体},就是指电导率很高,可以近似为无穷大的物质,例如银、铜、铝等金属
\begin{itemize}
    \item 理想导体的电导率为无穷大,因而边界面上可以存在面电流分布。
    \item 理想导体内部存在自由电荷,因而边界面上可以存在面电荷分布。
\end{itemize}
\begin{BoxFormula}[理想导体的边界条件]
    理想导体的边界条件满足
    \begin{Gather}[4pt]
        \vb*{e}_\text{n}\times\vb*{H}=\vb*{J}_S\\
        \vb*{e}_\text{n}\times\vb*{E}=\vb*{0}\\
        \vb*{e}_\text{n}\cdot\vb*{B}=0\\
        \vb*{e}_\text{n}\cdot\vb*{D}=\rho_S
    \end{Gather}
\end{BoxFormula}
理想导体内部还有许多重要的性质,由于理想导体的电导率$\sigma$为无穷大,根据$\vb*{J}=\sigma\vb*{E}$,这表明电场强度$\vb*{E}$应恒为零,否则将出现无限大的电流分布,因此,\empx{理想导体中不存在电场}。而进一步,我们由电磁感应原理$\curl\vb*{E}=-\pdv*{\vb*{B}}{t}=\vb*{0}$可知,\empx{理想导体中亦不存在时变磁场}。

\subsection{理想介质的边界条件}
理想介质,就是指电导率很低,可以近似为零的物质,例如聚苯乙烯、陶瓷
\begin{itemize}
    \item 理想介质的电导率为零,因而边界面上不存在面电流分布。
    \item 理想介质内部没有自由电荷,因而边界面上也不可能存在自由面电荷分布。
\end{itemize}
理想介质边界面上可以有束缚面电荷分布,但这不是$\div\vb*{D}=\rho$所关心的。
\begin{BoxFormula}[理想导体的边界条件]*
    理想导体的边界条件满足
    \begin{Gather}[4pt]
        \vb*{e}_\text{n}\times(\vb*{H}_1-\vb*{H}_2)=\vb*{0}\\
        \vb*{e}_\text{n}\times(\vb*{E}_1-\vb*{E}_2)=\vb*{0}\\
        \vb*{e}_\text{n}\cdot(\vb*{B}_1-\vb*{B}_2)=0\\
        \vb*{e}_\text{n}\cdot(\vb*{D}_1-\vb*{D}_2)=0
    \end{Gather}
\end{BoxFormula}