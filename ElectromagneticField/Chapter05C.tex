\section{电磁波的偏振}
根据\fancyref{fml:理想介质中平面电磁波的波函数}
\begin{Equation}
    \dot{E_x}(z)=E_{x\text{m}}\e^{\j\phi_x}\e^{-\j kz}\qquad
    \dot{E_y}(z)=E_{y\text{m}}\e^{\j\phi_y}\e^{-\j kz}
\end{Equation}
如果改写为时变形式
\begin{Equation}
    E_x(z,t)=E_{x\text{m}}\cos(\omega t-kz+\phi_x)\qquad
    E_y(z,t)=E_{y\text{m}}\cos(\omega t-kz+\phi_y)
\end{Equation}
而一般来说,两个同频正弦函数在垂直方向上叠加的轨迹是椭圆(试想椭圆的参数方程)。换言之,取决于$\phi_x$和$\phi_y$的关系,电场$\vb*{E}=\vb*{e}_xE_x+\vb*{e}_yE_y$的大小和方向可能会随时间变化,更直白的说,\empx{电场的振动是二维的振动},而二维上的振动不仅限于那种,矢量端点在一条直线上的往复运动,矢量端点沿圆轨迹或椭圆轨迹周期性旋转也是一种振动。我们将这种特性称为\uwave{偏振}(Polarization),前述的这几种振动模式,依次称为线偏振、圆偏振、椭圆偏振,其中椭圆偏振是最一般的情况,线偏振和圆偏振均可以视为椭圆偏振的特例。在本节,我们将会首先推导最一般的椭圆偏振方程,在此基础上,将线偏振和圆偏振作为椭圆偏振的特殊情况引出。

由于偏振研究的主要是振动,因此不妨取$z=0$简化
\begin{Equation}
    E_x(z)=E_{x\text{m}}\cos(\omega t+\phi_x)\qquad
    E_y(z)=E_{y\text{m}}\cos(\omega t+\phi_y)
\end{Equation}

\subsection{椭圆偏振}
在本小节,我们将求出\uwave{椭圆偏振}(Elliptical Polarization)的轨迹方程,以及其模和辐角。
\begin{BoxFormula}[椭圆偏振]
    椭圆偏振的轨迹方程为
    \begin{Equation}
        \qquad\qquad
        \qty(\frac{x}{E_{x\text{m}}})^2-
        2
        \qty(\frac{x}{E_{x\text{m}}})
        \qty(\frac{y}{E_{y\text{m}}})
        \cos(\phi_x-\phi_y)+
        \qty(\frac{y}{E_{y\text{m}}})^2=
        \sin^2(\phi_x-\phi_y)
        \qquad\qquad
    \end{Equation}
    模的表达式为
    \begin{Equation}
        \abs{\vb*{E}(z,t)}=
        \sqrt{E_{x\text{m}}^2\cos^2(\omega t+\phi_x)+E_{y\text{m}}^2\cos^2(\omega t+\phi_y)}
    \end{Equation}
    辐角的表达式为
    \begin{Equation}
        \tan\alpha=\frac{E_{y\text{m}}\cos(\omega t+\phi_y)}{E_{x\text{m}}\cos(\omega t+\phi_x)}
    \end{Equation}
\end{BoxFormula}

\begin{Proof}
    这里模和辐角无非$r=\sqrt{x^2+y^2}$和$\tan\theta=y/x$,没有什么特别的,关键在于轨迹方程。

    首先列出两个方向的振动方程
    \begin{Equation}&[1]
        E_x(z)=E_{x\text{m}}\cos(\omega t+\phi_x)\qquad
        E_y(z)=E_{y\text{m}}\cos(\omega t+\phi_y)
    \end{Equation}
    整理
    \begin{Equation}&[2]
        \frac{x}{E_{x\text{m}}}=
        \cos(\omega t+\phi_x)\qquad
        \frac{y}{E_{y\text{m}}}=
        \cos(\omega t+\phi_y)
    \end{Equation}
    将\xrefpeq{2}的两式求平方和
    \begin{Equation}&[3]
        \qty(\frac{x}{E_{x\text{m}}})^2+
        \qty(\frac{y}{E_{y\text{m}}})^2=
        \cos^2(\omega t+\phi_x)+
        \cos^2(\omega t+\phi_y)
    \end{Equation}
    运用降幂公式$\cos\theta=(1+\cos 2\theta)/2$
    \begin{Equation}&[4]
        \qquad\qquad
        \qty(\frac{x}{E_{x\text{m}}})^2+
        \qty(\frac{y}{E_{y\text{m}}})^2=
        \frac{1}{2}
        \qty[1+\cos(2\omega t+2\phi_x)]+
        \frac{1}{2}
        \qty[1+\cos(2\omega t+2\phi_y)]
        \qquad\qquad
    \end{Equation}
    这一支到\xrefpeq{4}结束,下面我们开始另外一支的推导。

    将\xrefpeq{2}的两式相乘,并乘以$2\cos(\phi_x-\phi_y)$
    \begin{Equation}&[5]
        \qquad\qquad
        \qty(\frac{x}{E_{x\text{m}}})
        \qty(\frac{y}{E_{y\text{m}}})
        2\cos(\phi_x-\phi_y)=
        \qty[2\cos(\omega t+\phi_x)\cos(\omega t+\phi_y)]\cos(\phi_x-\phi_y)
        \qquad\qquad
    \end{Equation}
    运用积化和差公式$\cos\theta_1\cos\theta_2=\qty[\cos(\theta_1+\theta_2)+\cos(\theta_1-\theta_2)]/2$
    \begin{Equation}&[6]
        \qquad
        \qty(\frac{x}{E_{x\text{m}}})
        \qty(\frac{y}{E_{y\text{m}}})
        2\cos(\phi_x-\phi_y)=
        \qty[\cos(2\omega t+\phi_x+\phi_y)+\cos(\phi_x-\phi_y)]\cos(\phi_x-\phi_y)
        \qquad
    \end{Equation}
    将$\cos(\phi_x-\phi_y)$乘入方括号中
    \begin{Equation}&[7]
        \qquad
        \qty(\frac{x}{E_{x\text{m}}})
        \qty(\frac{y}{E_{y\text{m}}})
        2\cos(\phi_x-\phi_y)=
        \cos(2\omega t+\phi_x+\phi_y)
        \cos(\phi_x-\phi_y)+
        \cos^2(\phi_x-\phi_y)
        \qquad
    \end{Equation}
    再次运用积化和差
    \begin{Equation}&[8]
        \qquad
        \qty(\frac{x}{E_{x\text{m}}})
        \qty(\frac{y}{E_{y\text{m}}})
        2\cos(\phi_x-\phi_y)=
        \frac{1}{2}[\cos(2\omega t+2\phi_x)+\cos(2\omega t+2\phi_y)]+\cos^2(\phi_x-\phi_y)
        \qquad
    \end{Equation}
    现在,将\xrefpeq{4}减去\xrefpeq{8}
    \begin{Equation}*
        \qquad\qquad
        \qty(\frac{x}{E_{x\text{m}}})^2-
        2
        \qty(\frac{x}{E_{x\text{m}}})
        \qty(\frac{y}{E_{y\text{m}}})
        \cos(\phi_x-\phi_y)+
        \qty(\frac{y}{E_{y\text{m}}})^2=
        \sin^2(\phi_x-\phi_y)
        \qquad\qquad
    \end{Equation}
    这就是椭圆偏振的轨迹方程。
\end{Proof}

\subsection{线偏振}
\uwave{线偏振}(Linear Polarization)是椭圆偏振在$\phi_x,\phi_y$相位相差$0$或$\pi$时的特殊情形。
\begin{BoxFormula}[线偏振]
    若$\phi_x-\phi_y=0+2n\pi$,则为1,3象限的线偏振,满足
    \begin{Equation}
        y=+\frac{E_{y\text{m}}}{E_{x\text{m}}}x\qquad
        \tan\alpha=+\frac{E_{y\text{m}}}{E_{x\text{m}}}
    \end{Equation}
    若$\phi_x-\phi_y=\pi+2n\pi$,则为2,4象限的线偏振,满足
    \begin{Equation}
        y=-\frac{E_{y\text{m}}}{E_{x\text{m}}}x\qquad
        \tan\alpha=-\frac{E_{y\text{m}}}{E_{x\text{m}}}
    \end{Equation}
\end{BoxFormula}

\begin{Proof}
    根据\fancyref{fml:椭圆偏振},椭圆偏振的轨迹方程为
    \begin{Equation}&[1]
        \qquad\qquad
        \qty(\frac{x}{E_{x\text{m}}})^2-
        2
        \qty(\frac{x}{E_{x\text{m}}})
        \qty(\frac{y}{E_{y\text{m}}})
        \cos(\phi_x-\phi_y)+
        \qty(\frac{y}{E_{y\text{m}}})^2=
        \sin^2(\phi_x-\phi_y)
        \qquad\qquad
    \end{Equation}
    当$\phi_x-\phi_y=0+2n\pi$时,有$\cos(\phi_x-\phi_y)=+1$和$\sin^2(\phi_x-\phi_y)=0$
    \begin{Equation}
        \qty(\frac{x}{E_{x\text{m}}}-\frac{y}{E_{y\text{m}}})^2=0\qquad
        y=+\frac{E_{y\text{m}}}{E_{x\text{m}}}x
    \end{Equation}
    当$\phi_x-\phi_y=\pi+2n\pi$时,有$\cos(\phi_x-\phi_y)=-1$和$\sin^2(\phi_x-\phi_y)=0$
    \begin{Equation}
        \qty(\frac{x}{E_{x\text{m}}}+\frac{y}{E_{y\text{m}}})^2=0\qquad
        y=-\frac{E_{y\text{m}}}{E_{x\text{m}}}x
    \end{Equation}
    而相应$\tan\alpha$的表达式很容易由上两式得到。
\end{Proof}

\subsection{圆偏振}
圆偏振(Circular Polarization)是椭圆偏振在$\phi_x,\phi_y$相位相差$\pi/2$或$3\pi/2$,且两个方向的振幅$E_{x\text{m}}=E_{y\text{m}}$时的特殊情况,若$E_{x\text{m}}\neq E_{y\text{m}}$,则为相应的正椭圆偏振(即长短轴为$x,y$轴)。

\begin{BoxFormula}[圆偏振]
    若$\phi_x-\phi_y=(\pi/2)+2n\pi\hphantom{3}$,且$E_{x\text{m}}=E_{y\text{m}}$,则为逆时针的右旋圆偏振,满足
    \begin{Equation}
        \qty(\frac{x}{E_{x\text{m}}})^2+
        \qty(\frac{y}{E_{y\text{m}}})^2=1
        \qquad
        \alpha=+(\omega t+\phi_x)
    \end{Equation}
    若$\phi_x-\phi_y=(3\pi/2)+2n\pi$,且$E_{x\text{m}}=E_{y\text{m}}$,则为顺时针的左旋圆偏振,满足
    \begin{Equation}
        \qty(\frac{x}{E_{x\text{m}}})^2+
        \qty(\frac{y}{E_{y\text{m}}})^2=1
        \qquad
        \alpha=-(\omega t+\phi_x)
    \end{Equation}
\end{BoxFormula}

\begin{Proof}
    根据\fancyref{fml:椭圆偏振},椭圆偏振的轨迹方程为
    \begin{Equation}&[1]
        \qquad\qquad
        \qty(\frac{x}{E_{x\text{m}}})^2-
        2
        \qty(\frac{x}{E_{x\text{m}}})
        \qty(\frac{y}{E_{y\text{m}}})
        \cos(\phi_x-\phi_y)+
        \qty(\frac{y}{E_{y\text{m}}})^2=
        \sin^2(\phi_x-\phi_y)
        \qquad\qquad
    \end{Equation}
    无论$\phi_x-\phi_y$是$(\pi/2)+2n\pi$还是$(3\pi/2)+2n\pi$,都有$\cos(\phi_x-\phi_y)=0$和$\sin^2(\phi_x-\phi_y)=1$
    \begin{Equation}
        \qty(\frac{x}{E_{x\text{m}}})^2+
        \qty(\frac{y}{E_{y\text{m}}})^2=1
    \end{Equation}
    根据\fancyref{fml:椭圆偏振},辐角为
    \begin{Equation}
        \tan\alpha=\frac{E_{y\text{m}}\cos(\omega t+\phi_y)}{E_{x\text{m}}\cos(\omega t+\phi_x)}
    \end{Equation}
    考虑到圆偏振时$E_{x\text{m}}=E_{y\text{m}}$
    \begin{Equation}
        \tan\alpha=\frac{\cos(\omega t+\phi_y)}{\cos(\omega t+\phi_y)}
    \end{Equation}\goodbreak
    若$\phi_x-\phi_y=(\pi/2)+2n\pi\hphantom{1}$,则$\cos(\omega t+\phi_y)=\cos(\omega t+\phi_x-\pi/2)\hphantom{3}=+\sin(\omega t+\phi_x)$
    \begin{Equation}
        \qquad\qquad
        \alpha=\arctan\qty[+\frac{\sin(\omega t+\phi_x)}{\cos(\omega t+\phi_y)}]=\arctan[+\tan(\omega t+\phi_x)]=+(\omega t+\phi_x)
        \qquad\qquad
    \end{Equation}
    若$\phi_x-\phi_y=(3\pi/2)+2n\pi$,则$\cos(\omega t+\phi_y)=\cos(\omega t+\phi_x-3\pi/2)=-\sin(\omega t+\phi_x)$
    \begin{Equation}
        \qquad\qquad
        \alpha=\arctan\qty[+\frac{\sin(\omega t+\phi_x)}{\cos(\omega t+\phi_y)}]=\arctan[-\tan(\omega t+\phi_x)]=-(\omega t+\phi_x)
        \qquad\qquad
    \end{Equation}
    这就得到了两种情况下辐角$\alpha$的表达式,这决定了圆偏振是右旋还是左旋。
\end{Proof}