\section{电磁场的位函数}
在静态电磁场中,我们曾引入电势(电位)和磁矢势(磁矢位)分别描述电场和磁场。而在时变电磁场中我们仍然可以沿用这种想法,引入电磁场的标量位和矢量位,以简化问题的分析。

\subsection{电磁场的矢量位与标量位}
在时变电磁场中,磁场仍然是无散的,因此,电磁场的矢量位的定义与磁矢势完全相同。
\begin{BoxDefinition}[电磁场的矢量位]
    定义电磁场的矢量位为
    \begin{Equation}
        \vb*{B}=\curl\vb*{A}
    \end{Equation}
\end{BoxDefinition}
在时变电磁场中,主要问题出在,电场不再是无旋的,即
\begin{Equation}
    \curl\vb*{E}=-\pdv{\vb*{B}}{t}
\end{Equation}
因此,我们就不再能像\fancyref{def:电势}那样定义标量位。

而为了解决这个问题,我们考虑在上式中代入\fancyref{def:电磁场的矢量位}
\begin{Equation}
    \curl\vb*{E}=-\pdv{t}\qty(\curl\vb*{A})
\end{Equation}
即
\begin{Equation}
    \curl(\vb*{E}+\pdv{\vb*{A}}{t})=\vb*{0}
\end{Equation}
由此可见,在时变电磁场中,尽管$\vb*{E}$不再无旋,但$\vb*{E}+\pdv*{\vb*{A}}{t}$是无旋的,以此定义标量位。
\begin{BoxDefinition}[电磁场的标量位]
    定义电磁场的标量位为
    \begin{Equation}
        \vb*{E}+\pdv{\vb*{A}}{t}=-\grad\varphi
    \end{Equation}
\end{BoxDefinition}
由此,电场强度$\vb*{E}$和磁感强度$\vb*{B}$就可以用位函数$\vb*{A},\varphi$表示了
\begin{Equation}
    \vb*{B}=\curl\vb*{A}\qquad
    \vb*{E}=-\pdv{\vb*{A}}{t}-\grad\varphi
\end{Equation}
现在的问题在于,由\fancyref{def:电磁场的矢量位}和\fancyref{def:电磁场的标量位}定义的矢量位$\vb*{A}$和标量位$\varphi$并不是唯一的,还存在另外的$\vb*{A}'$和$\vb*{\varphi}'$,使得
\begin{Equation}
    \vb*{B}=\curl\vb*{A}'\qquad
    \vb*{E}=-\pdv{\vb*{A}'}{t}-\grad\varphi'
\end{Equation}
实际上,设$\psi$为任意标量函数,令
\begin{Equation}
    \vb*{A}'=\vb*{A}+\grad\psi\qquad
    \varphi'=\varphi-\pdv{\psi}{t}
\end{Equation}
对于磁场部分,应用\fancyref{ppt:标量场的梯度无旋}
\begin{Equation}
    \vb*{B}'=\curl(A+\grad\psi)=\curl\vb*{A}+\curl(\grad\psi)=\curl\vb*{A}=\vb*{B}
\end{Equation}
对于电场部分
\begin{Equation}
    \vb*{E}'=-\pdv{\vb*{A}'}{t}-\grad\varphi'=-\pdv{t}\qty(\vb*{A}+\grad\psi)-\grad(\varphi-\pdv{\psi}{t})
\end{Equation}
稍作整理
\begin{Equation}
    \vb*{E}'=-\pdv{\vb*{A}}{t}-\grad\pdv{\psi}{t}-\pdv\varphi+\grad\pdv{\psi}{t}
\end{Equation}
即
\begin{Equation}
    \vb*{E}'=-\pdv{\vb*{A}}{t}-\pdv\varphi=\vb*{E}
\end{Equation}
由于$\psi$为任意标量函数,这会使$\vb*{A},\psi$有无穷多组,为此,我们需要额外规定矢量位$\vb*{A}$的散度。不同于静磁场的\fancyref{def:库伦规范},在电磁场中,使用的是以下的洛伦兹规范
\begin{BoxDefinition}[洛伦兹规范]
    在电磁场中,常令矢量位遵循\uwave{洛伦兹规范}(Lorenz Gauge)
    \begin{Equation}
        \div\vb*{A}=-\mu\varepsilon\pdv{\varphi}{t}
    \end{Equation}
\end{BoxDefinition}

\subsection{达朗贝尔方程}
达朗贝尔方程,就是用位函数表示的电磁场的波动方程。

\begin{BoxEquation}[达朗贝尔方程]
    电磁场中,达朗贝尔方程描述了位函数满足的波动方程
    \begin{Gather}[10pt]
        \laplacian\vb*{A}-\mu\varepsilon\pdv[2]{\vb*{A}}{t}=-\mu\vb*{J}\xlabelpeq{A}\\ 
        \laplacian\varphi-\mu\varepsilon\pdv[2]{\varphi}{t}=-\varepsilon^{-1}\rho\xlabelpeq{B}
    \end{Gather}
\end{BoxEquation}
\begin{Proof}
    \paragraph{矢量位部分}
    根据\fancyref{eqt:麦克斯韦方程组}的\xrefpeq[麦克斯韦方程组]{磁场旋度}
    \begin{Equation}&[1]
        \curl\vb*{H}=\vb*{J}+\varepsilon\pdv{\vb*{E}}{t}
    \end{Equation}
    运用本构关系$B=\mu\vb*{H}$代换
    \begin{Equation}&[2]
        \curl\vb*{B}=\mu\vb*{J}+\mu\varepsilon\pdv{\vb*{E}}{t}
    \end{Equation}
    根据\fancyref{def:电磁场的矢量位}和\fancyref{def:电磁场的标量位}
    \begin{Equation}&[3]
        \curl(\curl\vb*{A})=\mu\vb*{J}+\mu\varepsilon\pdv{t}\qty(-\pdv{\vb*{A}}{t}-\grad\varphi)
    \end{Equation}
    即
    \begin{Equation}&[4]
        \curl(\curl\vb*{A})=\mu\vb*{J}-\mu\varepsilon\pdv[2]{\vb*{A}}{t}-\mu\varepsilon\grad\pdv{\varphi}{t}
    \end{Equation}
    应用\fancyref{fml:矢量拉普拉斯的转化}
    \begin{Equation}
        \grad(\div\vb*{A})-\laplacian\vb*{A}=\mu\vb*{J}-\mu\varepsilon\pdv[2]{\vb*{A}}{t}-\mu\varepsilon\grad\pdv{\varphi}{t}
    \end{Equation}
    移项
    \begin{Equation}
        \laplacian\vb*{A}-\mu\varepsilon\pdv[2]{\vb*{A}}{t}-\grad(\div\vb*{A})-\mu\varepsilon\grad\pdv{\varphi}{t}=-\mu\vb*{J}
    \end{Equation}
    整理,合并梯度项
    \begin{Equation}
        \laplacian\vb*{A}-\mu\varepsilon\pdv[2]{\vb*{A}}{t}-\grad\qty[\div\vb*{A}+\mu\varepsilon\pdv{\varphi}{t}]=-\mu\vb*{J}
    \end{Equation}
    根据\fancyref{def:洛伦兹规范},即$\div\vb*{A}=-\mu\varepsilon\pdv*{\varphi}{t}$
    \begin{Equation}
        \laplacian\vb*{A}-\mu\varepsilon\pdv[2]{\vb*{A}}{t}=-\mu\vb*{J}
    \end{Equation}
    这就得到了\xrefpeq{A}。

    \paragraph{标量位部分}
    根据\fancyref{eqt:麦克斯韦方程组}的\xrefpeq[麦克斯韦方程组]{电场散度}
    \begin{Equation}
        \div\vb*{E}=\frac{\rho}{\varepsilon}
    \end{Equation}
    根据\fancyref{def:电磁场的标量位}
    \begin{Equation}
        \div(-\pdv{\vb*{A}}{t}-\grad\varphi)=\frac{\rho}{\varepsilon}
    \end{Equation}
    稍作整理
    \begin{Equation}
        \laplacian\varphi+\pdv{t}(\div\vb*{A})=\varepsilon^{-1}\rho
    \end{Equation}
    根据\fancyref{def:洛伦兹规范},即$\div\vb*{A}=-\mu\varepsilon\pdv*{\varphi}{t}$
    \begin{Equation}
        \laplacian\varphi-\mu\varepsilon\pdv[2]{\varphi}{t}=\varepsilon^{-1}\rho
    \end{Equation}
    这就得到了\xrefpeq{B}。
\end{Proof}
由此可以看出\fancyref{def:洛伦兹规范}的妙处,它显著简化了标量位和矢量位的波动方程。